\documentclass[upright, contnum]{umemoria}
\deptoA{DEPARTAMENTO DE INGENIERÍA ELÉCTRICA}
\deptoB{DEPARTAMENTO DE CIENCIAS DE LA COMPUTACIÓN}
\author{Matías Fernando Pavez Bahamondes}
\title{Diseño e Implementación de Memoria de Largo Plazo para Robots de Servicio}
\auspicio{}
\date{2018}
\guia{Javier Ruiz del Solar}
\carreraA{INGENIERO CIVIL ELÉCTRICO E}
\carreraB{INGENIERO CIVIL EN COMPUTACIÓN}
\memoria{MEMORIA PARA OPTAR AL TÍTULO DE INGENIERO CIVIL ELÉCTRICO E INGENIERO CIVIL EN COMPUTACIÓN}
\comision{JOCELYN SIMMONDS WAGEMANN, TODO 2, TODO 3}

\usepackage[T1]{fontenc}
\usepackage[spanish]{babelbib}
\usepackage{enumitem} % opciones para itemize 

% ------------- Imágenes -------------------------------------- %
\usepackage[pdftex]{graphicx}	% Para Archivos gráficos         %
\usepackage{float}				% Para usar H!                   %
\usepackage[section]{placeins}	% auto \FloatBarrier por sección %
\usepackage{caption}			% [font=small,labelfont=bf]      %
\usepackage{subcaption}         % Caption para subfiguras        %
\usepackage{sidecap}            % Captions al lado               %
\usepackage{wrapfig}            % Escribir alrededor             %
\graphicspath{{./figures/}}     % Agrega Path para buscar imgs.  %
% ------------------------------------------------------------- %

% = = = = = = = = = = = = = = = = = = = = = = =
% TODO NOTES
% = = = = = = = = = = = = = = = = = = = = = = =
\usepackage[pdftex,dvipsnames,table]{xcolor}  % Coloured text etc.
\usepackage{xargs} % Use more than one optional parameter in a new commands
%\usepackage[colorinlistoftodos,prependcaption,textsize=small]{todonotes}
\usepackage[colorinlistoftodos,prependcaption,textsize=tiny,disable]{todonotes}
\newcommandx{\unsure}[2][1=]{\todo[linecolor=red,backgroundcolor=red!25,bordercolor=red,#1]{#2}}
\newcommandx{\change}[2][1=]{\todo[linecolor=blue,backgroundcolor=blue!25,bordercolor=blue,#1]{#2}}
\newcommandx{\info}[2][1=]{\todo[linecolor=OliveGreen,backgroundcolor=OliveGreen!25,bordercolor=OliveGreen,#1]{#2}}
\newcommandx{\improvement}[2][1=]{\todo[linecolor=Plum,backgroundcolor=Plum!25,bordercolor=Plum,#1]{#2}}
\newcommandx{\thiswillnotshow}[2][1=]{\todo[disable,#1]{#2}}
\setlength{\marginparwidth}{0.5in}
% = = = = = = = = = = = = = = = = = = = = = = =


\hypersetup{
	colorlinks,
	linkcolor={black},
	citecolor={blue!50!black},
	urlcolor={blue!80!black}
}

%Para la Carta Gantt
\usepackage{gantt}
\usepackage{tikz}

\definecolor{Gray}{gray}{0.9}


\usepackage{lipsum}

\begin{document}

\frontmatter
\maketitle

\begin{abstract}
TO DO
\end{abstract}

\begin{dedicatoria}
TO DO: Una dedicatoria corta
\end{dedicatoria}

\begin{thanks}
TO DO
\end{thanks}

\cleardoublepage
\tableofcontents
\cleardoublepage
\listoftables
\cleardoublepage
\listoffigures

\mainmatter

\chapter{Introducción}\label{chapter:introduction}

\todo{INTRO: MIGRAR DESDE PREINFORME}
\todo[inline]{Intro: Descripción del capítulo}

\section{Antecedentes Generales}

\todo[inline]{Intro: Descripción de antecedentes}

\subsection{Robots de servicio domésticos}
% - 
% - 
% - 

\subsection{Equipo de trabajo: UChile Homebreakers}
% - 
% - 
% - 

\subsection{La memoria humana}
% - 
% - 
% - 


\section{Motivación}

\todo[inline]{Intro: Importancia de la memoria a largo plazo}

\subsection{Problema}
% - Implementación anterior de memoria a largo plazo
%    - No integrada correctamente
%    - Muy limitada
%    - Requiere de mucha intervención del equipo!. No mantenida
% - No existen software alternativos para resolver este problema.

\subsection{Oportunidad}
% - Diseño basado en requerimientos episódicos (cumplirlos todos)
% - Servidor genérico que no fuerce implementaciones y no limitado a sólo 1 robot
% - Arquitectura no invasiva y mantenible
% - Capacidad de integración a futuro en módulos de inferencia


\section{Objetivos del Proyecto}

\subsection{Objetivo general}
% - 
% - 
% - 

\subsection{Objetivos específicos}
% - 
% - 
% - 

\subsection{Alcances y contribución del trabajo}
% - 
% - 
% - 


\section{Estructura de la Memoria}
% - 
% - 
% - 
\chapter{Marco Teórico}\label{chapter:theory}

En este capítulo se revisan los temas y conceptos relevantes para el desarrollo del trabajo. Se formaliza la definición de robot doméstico y sus alcances. Se describe la memoria humana, sus categorías, funcionamiento y procesos cerebrales relevantes. Basado en los temas anteriores, se revisa la relación entre robótica y la memoria humana: se describen algunos enfoques existentes y las reglas generales para su implementación.


%% =====================================================================
\section{Robots de servicio domésticos}\label{sec:domestic_robots}
%% =====================================================================

La Federación Internacional de Robótica (IFR) \cite{IFR} define \textit{robot} como:
\begin{quotation}
	``Un mecanismo actuado y programable en dos o más ejes y con un cierto grado de autonomía, que se mueve en su entorno para realizar tareas previstas. En este contexto, autonomía se refiere a la habilidad de realizar tareas previstas, basado en el estado actual y lo sensado, sin intervención humana.''
\end{quotation}

Asimismo, la IFR define un \textit{robot de servicio} como un robot ``que realiza tareas útiles para humanos o equipamiento, excluyendo aplicaciones de automatización industrial''. Así, un robot de servicio debe trabajar en ambientes no controlados y con la autonomía suficiente que le permita llevar a cabo su cometido. Generalmente, la robótica de servicio se enfoca en asistir a los seres humanos en tareas repetitivas y comunes.

Según su área de aplicación, un robot de servicio se clasifica en \textit{de uso personal} o \textit{de uso profesional}. Los primeros son utilizados en ambientes no comerciales y por personas comunes; como por ejemplo, un robot sirviente o una silla de ruedas autónoma. Un robot de servicio profesional se utiliza en ambientes comerciales, usualmente operados por alguien entrenado; un ejemplo son los robots de entrega de paquetes o para cirugía.


Según la recopilación de datos realizada por la IFR durante el 2016, este tipo de robots es utilizado en las siguientes áreas:
\begin{itemize}[topsep=0pt]
	\setlength\itemsep{0.2em}
	\item Tareas domésticas: De compañía, asistencia, limpieza, cuidado del hogar.
	\item Entretenimiento: Juguetes, comunicación, educación e investigación.
	\item Asistencia a ancianos y discapacitados: Sillas robóticas y robots para cuidar personas.
	\item Transporte.
	\item Seguridad y vigilancia.
	\item Otros que no caen en las categorías anteriores.
\end{itemize}
\bigskip

El foco de este trabajo son los robots de servicio personales, dedicados a tareas domésticas, clasificación a la que en  adelante se referirá como \textit{Robots Domésticos}.

Para entender el alcance del trabajo, en cuanto a qué es lo que se espera del sistema, a continuación se listan algunas capacidades de los robots domésticos. Un robot de compañía y asistencia tiene, pero no se limita a las siguientes tareas:
\begin{itemize}[topsep=0pt]
	\setlength\itemsep{0.2em}
	\item Interacción amistosa con humanos.
	\item Ayudar a recordar y organizar tareas.
	\item Cooperar con la realización de un procedimiento.
	\item Guiar y seguir a personas.
	\item Recordar información y entidades.
\end{itemize}
\bigskip

Algunas tareas que robots domésticos de tipo mayordomo deben ejecutar son:
\begin{itemize}[topsep=0pt]
	\setlength\itemsep{0.2em}
	\item Ofrecer comida y bebestibles.
	\item Preparación de comida.\setcounter{tocdepth}{4}
	\item Ordenar y limpiar el hogar.
\end{itemize}
\bigskip

El primero de los objetivos específicos del proyecto tiene que ver con la derivación de consultas útiles para validar la memoria, a partir de las capacidades esperadas para un robot de servicio doméstico. En el Capítulo \ref{chapter:metodologia} se presenta cómo se planea desarrollar esta tarea.


%% =====================================================================
\section{Memoria humana}\label{sec:human_memory}
%% =====================================================================

\todoimprove{Poner referencias a todo esto. ... psicoanalisis ... ciencias cognitivas.}
% \cite{Deutsch2008} 

La memoria es un elemento fundamental para los humanos en su día a día, es parte integral de su existencia. Permite recordar quién, qué, cómo, dónde y cuándo. En términos psicológicos, es la habilidad para para codificar, almacenar y luego obtener información sobre eventos pasados, en el cerebro. Los pensamientos son parte de la memoria de corto plazo, mientras que eventos pasados son almacenados en una memoria de largo plazo. Existen muchos estudios en el área de la psicología cognitiva con diversas descripciones y modelos teóricos de cada tipo de memoria \cite{Vijayakumar2014}.

Desde el punto de vista de la información procesada, la memoria es vista como una facultad humana consistente en procesos para el manejo de información. Los 3 componentes principales son:

\begin{itemize}[topsep=0pt]
	\setlength\itemsep{0.2em}
	\item Codificación: En este paso, se adquiere nueva información desde los sentidos humanos. Los datos son convertidos a un formato que pueda ser almacenado en la estructura cerebral correspondiente.
	\item Almacenamiento: Consiste en la creación de registros permanentes de información. Es un proceso pasivo, de continuo procesamiento para clasificar datos nuevos y los ya existentes en el cerebro.
	\item Adquisición: Hace referencia al acceso de datos almacenados. El proceso se realiza en el momento, para obtener una reconstrucción aproximada de la información, a partir de elementos repartidos en distintas partes del cerebro.
\end{itemize}
%\bigskip


La memoria puede ser dividida en múltiples sistemas de independientes, con funcionalidades bien definidas y sustentados por distintas estructuras cerebrales. La primera diferenciación define dos tipos de memoria: la memoria de corto y la de largo plazo, STM (Short-Term Memory) y LTM (Long-Term Memory), por sus siglas en inglés. En el diagrama de la Figura \ref{img:human_memory} se muestra una separación clásica utilizada en el área de las ciencias cognitivas \cite{Eichenbaum:2008}, explicada en las siguientes subsecciones.

\usetikzlibrary{arrows,shapes,positioning,shadows,trees}
\tikzset{
	basic/.style  = {draw, drop shadow, font=\sffamily, rectangle},
	root/.style   = {basic, rounded corners=2pt, text width=10em, very thick, align=center, fill=gray!5},
	level 1/.style = {sibling distance=20mm},
	level 2/.style = {basic, rounded corners=6pt, thin,align=center, fill=red!10, text width=8em},
	level 3/.style = {basic, rounded corners=2pt, thin, align=center, fill=gray!10, text width=6.5em},
	level 4/.style = {basic, thin, align=left, fill=blue!10, text width=10em}
}

\begin{figure}[!h]
	\centering
	\begin{tikzpicture}[]
	
	\node [root] {Memoria Humana}
	child { node [level 2, xshift=-70pt] (c1) {\footnotesize Corto Plazo\\ (STM) }}
	child { node [level 2, xshift=100pt] (c2) {\footnotesize Largo Plazo\\ (LTM) }};
	
	\begin{scope}[every node/.style={level 3}]
	\node [below of = c2, xshift=-80pt, yshift=-20pt] (c21) {\footnotesize Explícita\\ (consciente)};  
	\node [below of = c2, xshift=80pt, yshift=-20pt] (c22) {\footnotesize Implícita\\ (inconsciente)};
	\end{scope} 
	
	\begin{scope}[every node/.style={level 4}]
	\node [below of = c21, xshift=40pt, yshift=-10pt] (c211) {\footnotesize Episódica (Ep-LTM)};
	\node [below of = c211, xshift=0pt, yshift=0pt] (c212) {\footnotesize Semántica (S-LTM)};
	
	\node [below of = c22, xshift=40pt, yshift=-10pt] (c221) {\footnotesize Procedural (P-LTM)};
	\node [below of = c221, xshift=0pt, yshift=0pt] (c222) {\footnotesize Primado};
	\node [below of = c222, xshift=0pt, yshift=0pt] (c223) {\footnotesize Emocional (Em-LTM)};
	\end{scope} 
	
	\draw[-, to path={-- (\tikztotarget)}]
	(c2) edge (c21)
	(c2) edge (c22);
	
	\draw[-, to path={|- (\tikztotarget)}]
	(c21.195) edge (c211.west)
	(c21.195) edge (c212.west);
	
	\draw[-, to path={|- (\tikztotarget)}]
	(c22.195) edge (c221.west)
	(c22.195) edge (c222.west)
	(c22.195) edge (c223.west);
	
	%
	
	%\draw[->, to path={-| (\tikztotarget)}]
	%  (c21) edge (c211) (c21) edge (c212);
	
	\end{tikzpicture}
	\caption{\small Clasificaciones de la memoria humana.}
	\label{img:human_memory}
\end{figure}


% Sobre el estudio de la memoria.. origenes.. Tulving 


\subsection{Memoria de corto plazo}
% - 
% - 
% - 
En el ámbito cognitivo, la STM se refiere a la habilidad de estar atento, recopilar información  y memorias, para luego utilizarlas dentro de un corto periodo de tiempo. Es responsable de almacenar información constantemente y de decidir que parte será transferida a la memoria de largo plazo. El término de \textit{Memoria de Trabajo} suele ser utilizado de manera intercambiable con el de STM.

La STM se caracteriza por manejar información muy detallada, ser de poca capacidad y permitir un rápido acceso a estos datos. Permite recordar rápidamente y con gran detalle experiencias ocurridas hace pocos segundos, pero con dificultad creciente a medida que avanza el tiempo.

Se sustenta principalmente en la corteza prefrontal del cerebro. Algunos estudios han mostrado que las neuronas involucradas son capaces de mantener información relevante de corto plazo, la que es combinada con información sensorial entrante y áreas que manejan la toma de decisiones. %(Miller, 2000).
En los humanos esta área presenta gran activación durante procesos de codificación, acceso y manipulación de memorias. %(Postle, 2006). 


\subsection{Memoria de largo plazo}

La LTM se asocia al almacenamiento permanente de información en el cerebro. Se caracteriza por manejar mucha información sobre experiencias y entidades, ser menos detallada y proveer un acceso más lento a los recuerdos, respecto a la STM \cite{Eichenbaum:2008}. Cierta información de la STM eventualmente es transferida a la LTM. De acuerdo a la Figura \ref{img:human_memory}, sus dos principales categorías son la \textit{Memoria Implícita} y la \textit{Memoria Explícita}.
% algunos creen que no está limitada en su capacidad de almacenar información.

\subsubsection{Memoria de largo plazo explícita}

La memoria explícita suele ser denominada \textit{memoria consciente} o \textit{memoria declarativa}, pues maneja conocimientos relacionados a hechos y eventos adquiridos de forma consciente. Según las estructuras cerebrales involucradas, se conforma de la \textit{memoria episódica} y de la \textit{memoria semántica}.

La memoria episódica (Ep-LTM) es de carácter  autobiográfico y almacena detalles de eventos y experiencias pasadas. Permite responder a las preguntas ``Qué sucedió'', ``Dónde ocurrió'' y ``Cuándo ocurrió''. Un humano puede acceder a esta memoria si es capaz de decir: ``recuerdo que''. Este tipo de memoria da al ser humano la sensación de continuidad en el tiempo.

La memoria semántica (S-LTM) almacena el conocimiento de hechos, significados, categorías y proposiciones. Un humano puede acceder a esta memoria si es capaz de decir: ``sé que''. Esta memoria se abstrae de perspectiva e información situacional.

Las estructuras cerebrales que soportan la memoria explícita son el hipocampo, encargado de manejar la Ep-LTM, junto a la corteza cerebral, en donde se distribuyen los conocimientos de la S-LTM. En el hipocampo se mantienen conexiones neuronales a los sectores de interés de la corteza, en donde se alojan conocimientos semánticos asociados a cada episodio.

Un ejemplo de uso de Ep-LTM es el recuerdo de una graduación escolar, el lugar y la fecha donde ocurrió. La S-LTM podría responder en que consiste una graduación y describir la ropa que se suele ocupar en ellas.


\subsubsection{Memoria de largo plazo implícita}

La memoria implícita  abarca la capacidad de aprender habilidades, hábitos y preferencias, caracterizados por ser mejorados o adquiridos sin una recolección consciente. Así, también suele ser denominada \textit{memoria inconsciente} o \textit{memoria no declarativa}, pues comprende acciones que pueden ser realizadas sin pensar en ellas. Ejemplos de esto, son el andar en bicicleta o tocar un instrumento musical.

Dos de sus componentes son la \textit{memoria procedural} (P-LTM) y la \textit{memoria de primado}. La primera ayuda a realizar tareas sin pensar en ellas, es decir, maneja el conocimiento del \textit{Cómo}; Ejemplos de esto son comer y caminar. La memoria de primado hace referencia a la predisposición para recordar hechos o información a la que un sujeto es expuesto con anterioridad; Ejemplos de esto son la facilidad para recordar canciones escuchadas hace poco tiempo, o el uso de palabras e ideas vistas recientemente.

Se ha mostrado que la P-LTM se sustenta en el cerebelo, mediante la activación de este durante el uso de habilidades motoras.

Un tercer componente de la memoria implícita es la \textit{memoria emocional} (Em-LTM). Se encarga de dar significado afectivo a ciertos  estímulos, que de otra forma serían neutrales. Las estructuras cerebrales involucradas son la amígdala, áreas corticales y subcorticales. Esta memoria se expresa mediante la activación del hipotálamo, en conjunto al sistema nervioso simpático, generando reacciones emocionales y sentimientos.


\subsection{Plasticidad sináptica y modulación}
% - 
% - 
% - 
Se denomina \textit{consolidación} de memoria al proceso de transición de conocimiento desde la STM a la LTM. Durante la consolidación se generan conexiones neuronales entre la Ep-LTM y la respectiva zona semántica. Para activar la consolidación se requiere de un estímulo relevante, sumado a la cadena de eventos para el almacenamiento.

Se denomina \textit{deterioro} de memoria u ``olvido'' al proceso de debilitamiento de las conexiones neuronales establecidas por los procesos de consolidación. Está en constante funcionamiento, degenerando las asociaciones entre la Ep-LTM y la S-LTM. Por lo tanto, en este contexto, el olvido no significa una eliminación de los datos en el cerebro, sino que estos siguen ahí, pero la conexión requerida es inexistente o es demasiado débil para poder ocuparla.

Existen procesos químicos a nivel cerebral que afectan la consolidación y el deterioro de la LTM. Hay evidencia de que estos están en continuo funcionamiento. Estos eventos celulares ocurren en una escala de segundos a minutos, y son esenciales para la mantención de la memoria a largo plazo.

Es posible modular ambos procesos. Las experiencias repetidas potencian la consolidación de la memoria, lo que fortalece las conexiones neuronales. Por otro lado, la memoria emocional es capaz de potenciar o deprimir las reacciones químicas requeridas; según los estímulos a los que se enfrente, modifica el nivel de relevancia de los eventos, pudiendo generar memorias muy fuertes y hábitos arraigados. Ejemplos de esto, son la memorización por repetición, los flashbacks y las memorias asociadas a eventos importantes como cumpleaños.


%% =====================================================================
\section{Memoria y robótica}\label{sec:robotic_memory}
%% =====================================================================

En esta sección se presenta el estado del arte respecto al uso de LTM en robótica. En primer lugar se presenta la importancia de la LTM y las expectativas para un robot doméstico. Luego se hace una comparación entre la memoria humana y el manejo de información en robots.  Se presenta el estado del arte para cada componente cognitivo de interés. Finalmente, se describen otros enfoques de la literatura para la implementación de LTMs, que no están basados en el enfoque biológico utilizado en este proyecto.


\subsection{Relevancia de la memoria robótica}
% - 
% - 
% - 
La memoria es una habilidad esencial para cualquier ser social. Lo mismo aplica para un robot doméstico cuya misión sea establecer una relación de largo plazo con usuarios humanos. Un problema común es que los usuarios tienden a perder el interés rápidamente en los robots, debido a la falta de vida y expectativas no cumplidas, respecto a la inteligencia y capacidad de socializar de la máquina. El problema se potencia con el paso del tiempo, donde la motivación por interactuar disminuye y se genera frustración, a medida que el robot continua repitiendo los mismos comportamientos predefinidos \cite{Ho2009}.

Si se desea mejorar la interacción humano-robot, entonces se requiere que el robot se comporte de manera más natural. Los mejores agentes robóticos sociales deberían satisfacer las necesidades cognitivas y sociales humanas; mientras más familiar sea la interacción, serán más efectivos en su propósito. Así, la LTM es una habilidad crucial si se espera que el robot sea capaz de aprender y adaptarse a su entorno.


Por otro lado, desde un punto de vista práctico, se ha mostrado que el concepto de memoria LTM aplicada a robots es beneficioso. En \cite{Salgado2012} ocupan memoria P-LTM para mejorar el desempeño de un robot en ambientes dinámicos, logrando acelerar el proceso de adaptación al entorno y la toma de decisiones.

%... blabla util \cite{Vijayakumar2014}



\subsection{Relación entre la memoria humana y la memoria robótica}
% - 
% - 
% - 
Son muchos los trabajos en LTM que han basado su desarrollo en la taxonomía de la memoria humana, donde se implementan esquemas de información con módulos análogos a los presentados en la Figura \ref{img:human_memory}. Esto se puede justificar por la similitud de cada tipo de memoria, con módulos preexistentes en la arquitectura robótica. A continuación se presenta una comparación entre cada tipo de memoria, sus procesos y el análogo robótico.


\paragraph{Memoria STM}
Se relaciona a todos los datos que están actualmente cargados en la memoria primaria de la máquina. Esta memoria es la utilizada para solucionar la tarea actual, es equivalente a los pensamientos del robot y cumple con las características de la STM humana: es volátil, de rápido acceso y limitada en capacidad. También se encuentra presente en todo archivo temporal manejado por el sistema, mientras está en funcionamiento. Así, la estructura equivalente a la cerebral sería principalmente la RAM de la máquina.

\paragraph{Memoria S-LTM}
La memoria S-LTM es común y se puede asociar a casi toda fuente de datos estática, no utilizada por las otras memorias. Luego, la S-LTM se sustenta en la memoria secundaria, cumpliendo las características de la LTM humana: es persistente, de acceso costoso y virtualmente ilimitada en capacidad. Algunos ejemplos son:
\begin{itemize}[topsep=0pt]
%	\setlength\itemsep{0.2em}
	\item Bases de datos.
	\item Directorios con imágenes de personas y objetos conocidos.
	\item Mapa con descripción del ambiente.
	\item Frases predefinidas que puede decir el robot.
	\item Archivos de audio utilizados por el robot.
	\item En general, todo archivo con datos persistentes, cargados en cada sesión de trabajo.
\end{itemize}


\paragraph{Memoria P-LTM}
Este tipo de memoria es comparable a algoritmos predefinidos para realizar acciones, generalmente motoras.  Algunos ejemplos comparables son: 
\begin{itemize}[topsep=0pt]
%	\setlength\itemsep{0.2em}
	\item Algoritmos basados en redes neuronales, entrenados para manipular objetos o reconocer patrones.
	\item Algoritmos entrenados para tareas específicas, cómo la detección de caras o el reconocimiento de voz.
	\item Controladores basados en puntos de operación para acciones motoras.
	\item Síntesis de voz.
\end{itemize}

Las estructuras equivalentes a la versión cerebral serían los archivos con parámetros para cada algoritmo, obtenidos a partir del entrenamiento o ajustados manualmente.


\paragraph{Otros tipos de memoria}
Generalmente, tanto STM, S-LTM como P-LTM son un requisito mínimo para el funcionamiento de un software robótico, por lo que no son implementadas de forma explícita, sino que se pueden identificar en los componentes de software descritos anteriormente. Luego, la existencia de tales memorias, no implica la intención de crear una arquitectura LTM similar a la humana. Los otros tipos de memorias sólo son implementados en casos especializados.



\subsection{Memoria LTM explícita}\label{sec:ltm_exp}
% - 
% - 
% - 
\todoimprove{Sobre approaches para crear memorias episódicas.}
%Han existido diversos intentos por crear memorias episódicas y semánticas. ... tal persona hizo tal cosa .... desde el 2000


\todoimprove{Sobre los diseños de consolidación utilizados en la literatura.. son pocos :(}

La memoria de mayor interés para este proyecto es la Ep-LTM, pues es la que permite generar interacciones humano-robot interesantes y que no sean repetitivas en el tiempo. Para su implementación, es un requisito disponer de S-LTM, pues la Ep-LTM almacena episodios y los cambios ocurridos a las entidades percibidas. La S-LTM almacena los modelos de cada entidad que llegarán a utilizar por la Ep-LTM.

No existe un consenso sobre los contenidos, el formato o las herramientas para implementar una Ep-LTM.
%... que recordar y cuando \cite{Kasap2010}
%... modelo de contexto \cite{Sanchez:2015} FELIX
% ... Considerar unificación de información.. que pasa si almaceno obj1 y obj2, pero luego aprendo que son el mismo?
%...... S Memory.. se abstrae de perspecttiva y cosas situacionales \cite{Stachowicz2012}
Sin embargo, si existe una aceptación generalizada sobre los requerimientos mínimos y deseables para el diseño \cite{Vijayakumar2014, Ho2009,  Stachowicz2012, Jockel2008}:

%\subsubsection{Aspectos de diseño}

\paragraph{Aspectos de diseño requeridos:}

% Stachowicz2012
% The first three requirements are given by the criteria of Clayton et al [14]

% corregirlos

\begin{enumerate}[topsep=0pt]
	\setlength\itemsep{0.2em}
	\item Contenido: (R1) La información de eventos pasados debe ser recolectada e indexada respecto a su contexto espacio-temporal: Qué, dónde y cuándo pasó.
	
	\item Estructura: (R2) Cada evento en conjunto con su contexto espacio-temporal forman una única representación integrada, que debe ser recordada como un todo, en caso de obtener cualquiera de las características del evento.
	
	\item Flexibilidad: (R3) La información almacenada es declarativa por naturaleza, y puede ser flexiblemente almacenada. Particularmente, puede interactuar con conocimiento semántico, incluso si este fue obtenido con posterioridad a la codificación del episodio.\todounsure{Está OK esta descripción?.. }
	
	\item Datos específicos: (R4) La memoria episódica cuenta con sólo una instancia de cada evento para su entrenamiento, pues cada evento tiene características específicas a la situación.
	
	\item Ep-LTM es LTM y declarativa: (R5) La memoria episódica es una forma de memoria LTM. Puede almacenar recuerdos por segundos, minutos, días o años. También es una forma de memoria declarativa; Es posible hablar sobre eventos asociados y acceder a ellos para introspección.\todounsure{Está OK esta descripción?}
	
	\item Perspectiva: (R6) La memoria episódica debe lidiar con datos específicos al evento, lo que implica una perspectiva. Es decir, eventos recordados deben mantener la misma perspectiva que se tenía en la experiencia original.
	
	\item Anidamiento: (R7) Los eventos almacenados en la memoria episódica pueden variar en tiempo y extensión. Particularmente, pueden ocurrir eventos dentro del actual.
	
	\item Trasposición : (R8) Los eventos almacenados en la memoria episódica pueden variar en tiempo y extensión. Particularmente, un evento A puede iniciar antes B, pero terminar durante la vida de B.
	
\end{enumerate}

\paragraph{Aspectos de diseño deseables:}

\begin{enumerate}[topsep=0pt]
	\setlength\itemsep{0.2em}
	\item No intrusivo: (R9) El campo ``Qué'' debe permitir almacenar información variable, organizada en estructuras de datos que no se conocen de antemano y que se ajustan a diversos módulos de procesamiento. Se espera que la LTM no requiera dependencias de módulos externos para poder funcionar y representar los datos. A la vez, no puede depender en que los otros módulos no cambien la representación de sus datos.
	\todoimprove{Reescribir este requisito.. La información variable no es requisito deseable, sino que obligatorio.. sólo la no intrusividad es deseable..}
	
	\item Eficiente: (R10) El sistema debe ser lo suficientemente eficiente para tolerar el manejo de una alta tasa de eventos, sin degradar el funcionamiento del robot. Es decir, todos los eventos deben ser procesados eventualmente, aún cuando el robot esté ocupando gran parte de sus recursos, y sin generar hambruna de CPU ni ancho de banda (de disco y red) al resto de los procesos.
	
	\item Escalable: (R11) Los costos asociados al manejo de la información (agregar, eliminar, actualizar y buscar datos) en la memoria deben escalar bien, respecto a la cantidad de datos almacenados. La memoria debe mantener los costos acotados, dentro de un rango que no entorpezca su uso.
	
\end{enumerate}



\subsubsection{Procesos de consolidación y deterioro}
%
%- algunos solo procuran desarrollar reglas sobre como actualizar los pesos de aprendizaje
%- dejar de aprender (sorprenderse al ser mayor)
%... modelo probabilistico para consolidar y decaer \cite{Dodd2005}
%... olvidar cosas.. si no se implementa, entonces la busqueda de informacion seria cada vez mas compleja. \cite{Deutsch2008}
%... reglas de consolidacion \cite{Dodd2005}

\todounsure{Hablar sobre este paper: sueño y postprocesamiento, puede estar demás para el marco teórico...}

En un esquema LTM, un episodio puede estar constituido de muchos eventos, pero no todos son igualmente relevantes. En su trabajo, Kelley \cite{Kelley2014} estudia 3 estrategias para la consolidación de recuerdos. La primera almacena todos los eventos ocurridos, pero tiene un costo de búsqueda lineal, respecto a los eventos almacenados; esta estrategia es impráctica a largo plazo. La segunda sólo almacena eventos interesantes y realiza una búsqueda entre los más recientes; esta estrategia es práctica, pero no permite abstracción del evento. La tercera estrategia se basa en un postprocesamiento de las memorias, de manera similar al sueño humano.

La estrategia propuesta por Kelley se basa en recordar todos los eventos, pero realizando un postprocesamiento de los datos una vez terminado el episodio. La ventaja es que no sólo permite recordar los eventos interesantes del episodio, sino que además permite reconocer pistas o estímulos previos que sirven para prevenir un evento indeseado o potenciar eventos interesantes. Además, permite almacenar eventos posteriores, que sirven para entender las consecuencias del evento de interés. 

En la Figura \ref{img:sleep_eventos} se muestra un episodio conformado de una secuencia de 9 eventos. Entre los marcadores 1-2 y 5-6 hay eventos considerados poco interesantes, mientras que los eventos entre 3-4 son interesantes. Kelley propone almacenar la secuencia completa de eventos, para descartar los que no son útiles en un postprocesamiento. A priori se deben quitar los eventos entre 1-2 y 5-6, sin embargo, el evento 2 se mantiene en la Ep-LTM como pista, y el evento 5 se mantiene para reforzar la consecuencia del episodio. Los resultados se pueden utilizar para aprendizaje reforzado, mientras que las pistas sirven para generalizar el episodio, en caso de que sean recurrentes.

\begin{figure}[!h]
	\centering
	\includegraphics[width=0.4\textwidth]{eventos.png}
	\caption{\small Ejemplo de una secuencia de eventos. Los cuadros coloreados y blancos indican eventos con poca o mucha relevancia, respectivamente. Los marcadores indican transiciones entre fases de poco y mucho interés. Obtenido de \cite{Kelley2014}.}
	\label{img:sleep_eventos}
\end{figure}


%Ejemplo:
%At some location (cue), something big and orange (Tiger) moved from left to right resulting in pain (event)



%Abstraccion de episodios de manera declarativa simbolica... permite mejorar tiempos de busqueda


Además, este diseño permite que el sistema cambie su opinión sobre una evento, mediante aprendizaje reforzado. Si se repiten eventos, pero la consecuencia deja de ser la misma, entonces el sistema se acostumbra.

%\subsubsection{Frameworks relacionados}
%
%... ISAC \cite{Dodd2005}
%... MINERVA, LIDA, Neuronal,M SMRTI \cite{Jockel2008}
%... deficiencias de ISAC, EPIROME, \cite{Stachowicz2012}
%... sobre Tecuci, ISAC, SOAR y Ho \cite{Deutsch2008}
%... definir contenido de QWHAT \cite{Stachowicz2012}
%.. definicion de episodio \cite{Dodd2005}
%... diseno explicado de SONIA en RDF \cite{Vijayakumar2014}

%\todo[inline]{Sobre la memoria procedural}
%\subsection{Memoria Procedural}
%
%... procedural y CRAM \cite{Winkler2014}
%\cite{Winkler2014}

%...... habilidades y PM \cite{Salgado2012}


\subsection{Memoria emocional}
% - 
% - 
% - 
La importancia de un evento se ve fuertemente influenciada por el estado emocional de una persona. Por lo tanto, la decisión de que almacenar o recordar depende de las emociones \cite{Deutsch2008}.

Su implementación requiere como mínimo de un mapeo entre estímulos percibidos por el robot y las sensaciones emocionales que estos generan. Dood et al. \cite{Dodd2005} propone el uso de la teoría emocional de reacciones de Haikonen, que considera a una emoción como una combinación de estímulos básicos. Las sensaciones elementales son: bienestar, malestar, dolor, placer e interés.

Dood et al. proponen implementar las sensaciones a partir de distintos estímulos medidos en un robot:

\begin{itemize}[topsep=0pt]
	\setlength\itemsep{0.2em}
	\item Actuador que se aproxima a sus límites de movimiento físico o de fuerza. 
	\item Nivel de iluminación percibido.
	\item Nivel de ruido acústico percibido.
	\item Ausencia o presencia de humanos. Falta de interacción.
	\item Cumplimiento de objetivos.
	\item Cumplimiento de expectativas.
\end{itemize}


Sistemas más avanzados, incluso pueden considerar la generación de reacciones emocionales, basándose en las sensaciones derivadas anteriormente. En la Figura \ref{img:emotional_haikonen} se muestran las reacciones generadas según el modelo de Haikonen. Además, estas se podrían reflejar en la personalidad del robot, por ejemplo, mediante gestos, vocabulario o nivel de aceptación para realizar una acción. Kasap et al. \cite{Kasap2010} utilizan un sistema llamado \textit{Emotion Engine}, para generar reacciones emocionales y simular cambios de personalidad de un robot, según las sensaciones percibidas.

\begin{figure}[!h]
	\centering
	\begin{tabular}{| l | l |}
		\hline
		\rowcolor{gray!50}
		Sensación Elemental & Reacción  \\ 
		\hline Bueno: gusto, aroma & Aceptación, Acercar \\ 
		\hline Malo: gusto, aroma & Rechazo, Alejar \\ 
		\hline Dolor: autoinfligido  & Alejar, Desistir \\ 
		\hline Dolor: agente externo & Agresión \\ 
		\hline Dolor: sobre esfuerzo & Sumisión \\ 
		\hline Placer & Mantener, Acercar \\ 
		\hline Acierto & Mantener atención \\ 
		\hline Desacierto & Migrar atención \\ 
		\hline Novedad & Enfocar atención \\ 
		\hline 
	\end{tabular} 
	\caption{\small Sensaciones elementales y sus reacciones correspondientes, según el modelo de Haikonen. Obtenido de \cite{Dodd2005}.}
	\label{img:emotional_haikonen}
\end{figure}


Para su uso efectivo dentro de un esquema LTM, se espera que las sensaciones reportadas incluyan un nivel de intensidad. Según el nivel percibido en cada episodio, es posible clasificarlos entre eventos muy o poco relevantes. Los más relevantes tendrán mayor probabilidad de ser recuperados al recordar. Deutsch et al.  \cite{Deutsch2008} consideran que el la intensidad de las sensaciones es importante, pues permite evitar costos de búsqueda lineales dentro de todos los episodios almacenados. Por otro lado, Dood et al. proponen curvas de decaimiento para la importancia de los episodios, que permiten simular la pérdida de interés en los eventos.


%... + refs sobre emociones \cite{Deutsch2008}


%\todo[inline]{Resumen de dificultades ... diagrama o tabla.}


\subsection{Otros enfoques}
% - 
% - 
% - 
A continuación se presentan algunos estudios relacionados con aspectos de una memoria Ep-LTM que escapan de los requerimientos para este proyecto o que simplemente no son basados en la taxonomía de la memoria humana. Estos trabajos sólo se presentan a modo de completitud, pues no permiten resolver el objetivo de este proyecto, sino que sólo comprender otros enfoques y acercamientos a la solución. 

El sistema propuesto por Ho et al. \cite{Ho2009} busca modelar la memoria de forma suficientemente general, como para permitir el traspaso de los recuerdos de un robot a otro, independientemente de que el hardware sea distinto; El costo de esto, es que se reduce la personalización de cada robot. Ho et al. además aplican la teoría \textit{Roboética}, sugerida por Veruggio y Operto \cite{Veruggio2006}, de donde derivan restricciones de diseño, relativas al manejo de información privada de los usuarios.

En \cite{KimMinJoo2016}, Kim et al. plantean el uso de Deep Learning para modelar la memoria episódica y la planificación de acciones de manera holística. En su implementación, los procesos de codificación, almacenado y recuperación de episodios son manejados como uno solo. Los procesos de decaimiento y relevancia son abstraídos, para ser manejados automáticamente por la red.

Thorsten et al. \cite{Spexard2008} proponen una memoria LTM para el robot BIRON. En su trabajo, se abstraen de la clasificación entre memorias Ep-LTM y S-LTM, pues todos los datos de largo plazo almacenados por el robot son considerados LTM. La memoria almacena sólo datos de alto nivel, obtenidos tras el procesamiento de streams de datos básicos, como cámaras, micrófonos o actuadores. Los datos almacenados corresponden a un historial de percepciones y acciones de alto nivel realizadas, como: detecciones de objetos, interacciones verbales o la descripción de movimientos realizados. A pesar de su simplicidad, esta arquitectura centralizada permite reducir el las dependencias entre si de cada componente y reducir el ancho de banda utilizado para retransmitir la información entre procesos.

\todoimprove{citar a \cite{Pratama2014}}




%% =============================================================================
\section{Revisión de sistemas LTM}
%% =============================================================================
% - es necesario mostrar otros "approaches" que no se utilizarán??
% - mostrar sistemas similares LTM?
% - 
.

\todowrite{SOBRE TRABAJO ANTERIOR EN BENDER: Memoria considerada no suficiente.}
\todounsure{es esto redundante?}



\chapter{Implementación}\label{chapter:implementacion}

\chapter{Resultados y Análisis}\label{chapter:results}
\begin{conclusion}

\todolater{Conclusión: Escribir al final}

\todowrite{Sobre paper escrito... conocimiento adquirido para promover LTM en @Home}
%También se considera que es la oportunidad de promover la inclusión de desafios basados en LTM en la liga @Home, a partir de los resultados de este trabajo. Así, el desarrollo de LTMs y capacidades asociadas dejaría de ser postergado y pasaría a ser una prioridad para los equipos de la competencia.

\todowrite{Sobre tiempo gastado en KnowRob, Prolog y Java.. app 2 meses.}

\section*{Trabajo Futuro}

\end{conclusion}
\begin{glosario}\label{chapter:glosario}

\todoimprove{Revisar \url{https://es.sharelatex.com/learn/Glossaries}}

\todounsure{Agregar cosas en cursiva?, cosas en inglés y siglas.}

\begin{itemize}
\item API
\item plugin
\item stream
\item LTM
\item launchfile
\end{itemize}

\end{glosario}	


\bibliographystyle{IEEEtran}
\bibliography{IEEEabrv,bibliography}
%\bibliographystyle{plain}
%\bibliography{bibliografia}

\end{document}
