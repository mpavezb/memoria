\usepackage[T1]{fontenc}
\usepackage[spanish]{babelbib}
\usepackage{enumitem} % opciones para itemize 

% ------------- Imágenes -------------------------------------- %
\usepackage[pdftex]{graphicx}	% Para Archivos gráficos         %
\usepackage{float}				% Para usar H!                   %
\usepackage[section]{placeins}	% auto \FloatBarrier por sección %
\usepackage{caption}			% [font=small,labelfont=bf]      %
\usepackage{subcaption}         % Caption para subfiguras        %
\usepackage{sidecap}            % Captions al lado               %
\usepackage{wrapfig}            % Escribir alrededor             %
\graphicspath{{./figures/}}     % Agrega Path para buscar imgs.  %
% ------------------------------------------------------------- %

% = = = = = = = = = = = = = = = = = = = = = = =
% TODO NOTES
% = = = = = = = = = = = = = = = = = = = = = = =
\usepackage[pdftex,dvipsnames,table]{xcolor}  % Coloured text etc.
\usepackage[colorinlistoftodos,prependcaption,textsize=small]{todonotes}
%\usepackage[colorinlistoftodos,prependcaption,textsize=tiny,disable]{todonotes}
\newcommand{\todoimplementation}[1]{\todo[linecolor=blue,backgroundcolor=blue!25,bordercolor=blue,inline]{CODE ME: #1}}
\newcommand{\todopaused}[1]{\todo[linecolor=red,backgroundcolor=red!25,bordercolor=red,inline]{PAUSED: #1}}
\newcommand{\todounsure}[1]{\todo[linecolor=green,backgroundcolor=green!25,bordercolor=green,inline]{UNSURE: #1}}
\newcommand{\todolater}[1]{\todo[linecolor=black,backgroundcolor=black!25,bordercolor=black,inline]{LATER: #1}}
\newcommand{\todofinal}[1]{\todo[linecolor=black,backgroundcolor=black!10,bordercolor=black,inline]{VERSION FINAL: #1}}
\newcommand{\todoimprove}[1]{\todo[linecolor=yellow,backgroundcolor=yellow!25,bordercolor=yellow,inline]{IMPROVEMENT: #1}}
\newcommand{\todowrite}[1]{\todo[linecolor=purple,backgroundcolor=purple!25,bordercolor=purple,inline]{WRITE ME: #1}}
\setlength{\marginparwidth}{0.5in} % mostrar correctamente todos al margen 
% = = = = = = = = = = = = = = = = = = = = = = =


\hypersetup{
	colorlinks,
	linkcolor={black},
	citecolor={blue!50!black},
	urlcolor={blue!80!black}
}

\usepackage{tikz}


%% Sections formatting
% -----------------------------------------

\usepackage{titlesec}
\usepackage[toc]{appendix}
\usepackage{etoolbox}
\makeatletter
% appendix and hyperref packages are broken with utf8!
% https://tex.stackexchange.com/questions/58848/ap%c3%a9ndices-appendix-spanish-accent
\appto{\appendices}{\def\Hy@chapapp{Appendix}}

% title sec is bugged on texlive-full deb. use this to correct it.
% Source: https://bugs.launchpad.net/ubuntu/+source/texlive-extra/+bug/1574052
\patchcmd{\ttlh@hang}{\parindent\z@}{\parindent\z@\leavevmode}{}{}
\patchcmd{\ttlh@hang}{\noindent}{}{}{}
\makeatother
% -----------------------------------------
% less spaced \paragraph sections
\titlespacing{\paragraph}{0pc}{0.0ex plus .1ex minus .2ex}{1pc}
%{12pc}{1.5ex plus .1ex minus .2ex}{1pc}

% Fancy chapters.. more on: https://ctan.org/pkg/fncychap
%Options: Sonny, Lenny, Glenn, Conny, Rejne, Bjarne, Bjornstrup, PetersLenny
\usepackage[Bjornstrup]{fncychap}

% Numbering and TOC for: 
% Chapters(0), Sections(1), subsections(2), subsubsections(3)
\setcounter{tocdepth}{2}
\setcounter{secnumdepth}{3}

% highlight
\usepackage{soul}
%\setulcolor{red}
%\sethlcolor{blue}
%\renewcommand\ul[1]{#1} % <<<<<<<<<<<<<<<<<<<<<<<<<<<<<<<<<<<<<
%\renewcommand\hl[1]{#1} % <<<<<<<<<<<<<<<<<<<<<<<<<<<<<<<<<<<<<


%%----- Colores --------------------------------------------------%
%\usepackage{color}												%
%\usepackage{colortbl}											%
%\usepackage[usenames,dvipsnames,svgnames,table]{xcolor}			%
%\definecolor{gray}{rgb}{0.51,0.51,0.51}							%
%\definecolor{dkgreen}{rgb}{0,0.6,0}								%
%\definecolor{mauve}{rgb}{0.58,0,0.82}							%
%%----------------------------------------------------------------%


% -------------------------------------------------------------------- %
% ----- LISTINGS ----------------------------------------------------- %
% -------------------------------------------------------------------- %
\usepackage{listingsutf8}

\renewcommand{\lstlistingname}{C\'odigo}
\renewcommand{\lstlistlistingname}{Listado de \lstlistingname s}

% --------------------------------------------------------%
% --------------- ESTILOS  -------------------------------%
% --------------------------------------------------------%
\lstdefinestyle{/Style/code}{
	%
	% ---- structure ----
	%title=\lstname,                    % show filename of files of title
	caption=\lstname,                   % also try caption instead
	captionpos=t,                       % caption-position to bottom
	basicstyle=\scriptsize\ttfamily,    % \footnotesize
	columns=fullflexible,               % %fixed
	tabsize=2,
	aboveskip={1.5\baselineskip},
	%
	% ---- numbers ----
	numbers=left,
	stepnumber=1,
	numbersep=5pt,                      % distance(line-numbers, code)
	numberstyle=\tiny\color{gray},      % for line-numbers	
	%
	% ---- shows ----
	showspaces=false,                   % spaces with particular underscores
	showstringspaces=false,             % underline spaces within strings
	showtabs=false,                     % tabs with particular underscores
	frame=false,                        % adds a frame around the code 
	%                                   % false, tb, shadowbox, single, lines
	%
	% ---- color ----
	backgroundcolor=\color{white},
	rulecolor=\color{black},
	%
	% ---- break ----
	breakatwhitespace=false,            % automatic breaks should only happen at whitespace
	breaklines=true,                    % automatic line breaking
	%                                   % Simbolo mostrado para el breakline
	prebreak = \raisebox{0ex}[0ex][0ex]{\ensuremath{\hookleftarrow}},
	%
	% ---- Extras ----
	%escapeinside={\%*}{*)},         	% LaTeX within your code
	extendedchars=true,
	inputencoding=utf8/latin1 			% Tildes
}

\lstdefinestyle{/Style/C}{
	language=C,
	keywordstyle=\bfseries\ttfamily\color[rgb]{0,0,1},
	identifierstyle=\ttfamily,
	commentstyle=\color[rgb]{0.133,0.545,0.133},
	stringstyle=\ttfamily\color[rgb]{0.627,0.126,0.941},
	style=/Style/code
}
\lstdefinestyle{/Style/C++}{
	language=C++,
	keywordstyle=\bfseries\ttfamily\color[rgb]{0,0,1},
	identifierstyle=\ttfamily,
	commentstyle=\color[rgb]{0.133,0.545,0.133},
	stringstyle=\ttfamily\color[rgb]{0.627,0.126,0.941},
	style=/Style/code
}

% --- XML de ROS ----------->
\lstdefinelanguage{/XML/ROS}{
	sensitive=false,
	morestring=[b]",
	%morestring=[s]{>}{<},
	morecomment=[s]{<!--}{-->},
	morecomment=[s]{<?}{?>},
	morekeywords={master, include, launch, node, param, rosparam, group, machine, arg} % list of attributes
}

\lstdefinestyle{/Style/XML/ROS}{
	language=/XML/ROS,
	stringstyle=\color{WildStrawberry},
	identifierstyle=\color{RoyalPurple},
	keywordstyle=\color{Emerald},
	%commentstyle=\color{Purple},	de los <? ?>
	commentstyle=\color{Blue},
	style=/Style/code
}
\lstdefinestyle{/Style/sh}{
	language=sh,
	stringstyle=\color{WildStrawberry},
	identifierstyle=\color{RoyalPurple},
	keywordstyle=\color{red},
	commentstyle=\color{Blue},
	style=/Style/code
}


% --- YAML de ROS ----------->
\newcommand\YAMLcolonstyle{\color{red}\mdseries}
\newcommand\YAMLkeystyle{\color{black}\bfseries}
\newcommand\YAMLvaluestyle{\color{blue}\mdseries}
\lstdefinelanguage{/YAML/ROS}{
    sensitive=false,
    morestring=[b]",
    morestring=[b]',
    moredelim=**[il][\YAMLcolonstyle{:}\YAMLvaluestyle]{:},
    morecomment=[l]{\#}
}
\lstdefinestyle{/Style/yaml/ROS}{
	language=/YAML/ROS,
	stringstyle=\color{Green},
	identifierstyle=\color{Black},
	keywordstyle=\color{Red},
	commentstyle=\color{Gray},
	style=/Style/code
}


% colores disponibles: https://en.wikibooks.org/wiki/LaTeX/Colors
% --- MSG de ROS ----------->
\lstdefinelanguage{/ROS/MSG}{
	%	alsodigit={-},
	alsoletter={},
	sensitive=true,
	comment=[l]{\#},
	keywords=[1]{
		uint32, uint8, string, float32, int32, bool, time,
		geometry\_msgs,Point,sensor\_msgs,Image
	},
	keywords=[2]{
		ltm, ltm\_samples, What, When, Where, Episode,
		Info, Date, Relevance, EmotionalRelevance, HistoricalRelevance
	},
	%otherkeywords={---,=}
}
\lstdefinestyle{/Style/ROS/MSG}{
	language=/ROS/MSG,
	stringstyle=\color{Green},
	identifierstyle=\color{Black},
	keywordstyle=[1]\bf\color{Red},
	keywordstyle=[2]\bf\color{Blue},
	commentstyle=\color{Gray},
	style=/Style/code
}

% custom caption for listings
\usepackage{caption}
\DeclareCaptionFont{listingfont}{\color{white}}
\DeclareCaptionFont{listinglabelfont}{\bf\color{white}}
\DeclareCaptionFormat{listing}{\colorbox{gray}{\parbox{\textwidth}{#1#2#3}}}
\captionsetup[lstlisting]{format=listing,labelfont=listinglabelfont,textfont=listingfont}
