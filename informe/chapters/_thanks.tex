\begin{dedicatoria}
A mi familia que me ha apoyado para crecer y convertirme en el que soy.
\vspace{50pt}
\begin{quote}
	``Las preguntas que no podemos contestar son las que más nos enseñan. Nos enseñan a pensar. Si le das a alguien una respuesta, lo único que obtiene es cierta información. Pero si le das una pregunta, él buscará sus propias respuestas. Así, cuando encuentre las respuestas, las valorará más. Cuanto más difícil es la pregunta, más difícil la búsqueda. Cuanto más difícil es la búsqueda, más aprendemos. Una pregunta imposible \ldots''
	%	“It’s the questions we can’t answer that teach us the most. They teach us how to think. If you give a man an answer, all he gains is a little fact. But give him a question and he’ll look for his own answers. That way, when he finds the answers, they’ll be precious to him. The harder the question, the harder we hunt. The harder we hunt, the more we learn. An impossible question …”
\end{quote}
\rightline{{\rm --- Patrick Rothfuss, El Temor de un Hombre Sabio}}
%\rightline{{\rm --- Patrick Rothfuss, The Wise Man's Fear}}

\end{dedicatoria}

\begin{thanks}

En primer lugar, agradezco a mi familia: padres, hermanos, tíos y abuelos, que siempre me han apoyado en este largo camino. Especialmente, me gustaría agradecer a mis ninos, Pamela y Mario, por todo el cariño que me han brindado siempre, por el espacio que me brindaron en su casa durante mis primeros años de universidad y porque me siempre me han querido como un hijo más.

Además, debo agradecer a mis amigos, los que siempre han estado ahí en las buenas y en las malas. Por las horas y horas dedicadas al HOTS, en sacar adelante al equipo. Por los muchos años de amistad, muchas gracias Michi, Claudio y Carla.

Gracias a todos quienes con que he trabajado en el equipo de robótica del DIE. Por todo el esfuerzo que le pusimos a nuestros robots. Gracias a quines me guiaron en los primeros años de trabajo y a quienes conocí en el proceso. Gracias a Bender.

Debo agradecer a la profesora Jocelyn Simmonds por sus minuciosas correcciones a este extenso informe, sumado a sus consejos que me permitieron sacar adelante este trabajo de título. También, agradezco al profesor Javier Ruiz del Solar y el resto de la comisión por su tiempo y correcciones.
%% - ... JR por darme más y más pega, sin proveer consejos útiles.

%% - Conductor de bus que me atropelló... para tener más tiempo para trabajar en la memoria
Finalmente, agradezco a mi pareja Luz, que me ha acompañado, ayudado y brindado su cariño durante estos últimos meses. Estamos iniciando una nueva etapa, con nuevas metas y desafíos, estoy feliz por poder compartirlos contigo. Muchas gracias lucy, te amo.

\end{thanks}
