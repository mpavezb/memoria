\chapter{Marco Teórico}\label{chapter:theory}

\todo{MARCO TEORICO: MIGRAR DESDE PREINFORME}



%% =============================================================================
\section{Memoria robótica a largo plazo}
%% =============================================================================

\subsection{Robots de servicio domésticos}
%% -----------------------------------------------------------------------------
% - 
% - 
% - 

\subsection{Memoria humana}
%% -----------------------------------------------------------------------------
% - 
% - 
% - 

\subsubsection{Memoria de corto plazo}
% - 
% - 
% - 

\subsubsection{Memoria de largo plazo}
% - 
% - 
% - 

\subsubsection{Plasticidad sináptica y modulación}
% - 
% - 
% - 
 
\subsection{Memoria y robótica}
%% -----------------------------------------------------------------------------
% - 
% - 
% - 

\subsubsection{Relevancia de la memoria robótica}
% - 
% - 
% - 

\subsubsection{Relación entre la memoria humana y la memoria robótica}
% - 
% - 
% - 

\subsubsection{Memoria LTM explícita}
% - 
% - 
% - 

\subsubsection{Memoria emocional}
% - 
% - 
% - 

\subsubsection{Otros enfoques}
% - 
% - 
% - 



%% =============================================================================
\section{Algoritmos y metodologías de programación}
%% =============================================================================
% - plugins
% - convex hull
% - observer
% - UDP/TCP
% - Algs utilizados para emociones implementadas en robot
% - TO DO: Estructura de arbol particular a utilizar
% - Sistemas de coordenadas?? Frames?
% - plutchik's wheel of emotions
% - haikonen feel-emotion-reaction theory (usado para diseñar emociones para el robot)
% - procesamiento de imágenes: degradación de streams


%% =============================================================================
\section{Componentes de software y hardware}
%% =============================================================================
% - 
% - 
% - 
TO DO

\subsection{MongoDB}
%% -----------------------------------------------------------------------------
% - Overview No relacional
% - Collections
% - Datos binarios
% - consultas

\subsection{ROS}
%% -----------------------------------------------------------------------------
% - overview
% - comunidad y packages
% - utilidad
% - python-c++ + otros


\subsubsection{Conceptos}
% - distribuciones
% - mensajes
% - tópicos
% - servicios
% - parametros
% - nodo
% - transporte de información: RED, tcp/udp 
% - package y dependencias

\subsubsection{Herramientas}
% - roslaunch
% - rviz
% - rosbag (probablemente se ocupará para test suite)
% - pluginlib: muy utilizada. dependencia confiable y mantención
% - warehouse_ros_mongo. Utilizada también por MoveIt, por dependencia confiable y mantención


\subsection{Smach}
%% -----------------------------------------------------------------------------
% - Overview
% - Sobre su uso común
% - Ejemplos de máquinas de estado y sus alcances

\subsection{UChile ROS Framework}
%% -----------------------------------------------------------------------------
% - copiar cosas del preinforme
% - módulos de interés
% - funcionando completamente en ROS. 
% - sobre módulo uchile_knowledge de bajo nivel

\subsection{Bender}
%% -----------------------------------------------------------------------------
% - sensores de interés
% - actuadores de interés
% - sobre cómo obtener datos desde el robot: API ROS



%% =============================================================================
\section{Revisión de sistemas LTM}
%% =============================================================================
% - es necesario mostrar otros "approaches" que no se utilizarán??
% - mostrar sistemas similares LTM?
% - 
\unsure[inline]{es esto redundante?}

