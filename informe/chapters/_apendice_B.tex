\chapter{Anexo: Implementación}\label{chapter:appendix_b}

En este capítulo se presentan bloques de código utilizados para la implementación del proyecto, los que complementan las explicaciones dadas en el Capítulo \ref{chapter:implementacion}. En primer lugar, se presentan todos los mensajes ROS que conforman la definición de un episodio. Luego, se presenta código utilizado para la API ROS del servidor LTM.

\section{Modelo de Datos}\label{appendixB:modelo_datos}

A continuación se muestran todos los mensajes que definen la estructura de un episodio, los que son embebidos en el mensaje \texttt{ltm/Episode.msg}.

\todoimprove{Reescribir comentarios para código de mensajes y servicios. Eliminar campos extra, para evitar llenar el informe de basura.}

\lstset{style=/Style/ROS/MSG}
\lstinputlisting[caption=ltm/When.msg]{code/msg/When.msg}
\lstinputlisting[caption=ltm/Where.msg]{code/msg/Where.msg}
\lstinputlisting[caption=ltm/What.msg]{code/msg/What.msg}
\lstinputlisting[caption=ltm/StreamRegister.msg]{code/msg/StreamRegister.msg}
\lstinputlisting[caption=ltm/EntityRegister.msg]{code/msg/EntityRegister.msg}
\lstinputlisting[caption=ltm/Info.msg]{code/msg/Info.msg}
\lstinputlisting[caption=ltm/Relevance.msg]{code/msg/Relevance.msg}
\lstinputlisting[caption=ltm/EmotionalRelevance.msg]{code/msg/EmotionalRelevance.msg}
\lstinputlisting[caption=ltm/HistoricalRelevance.msg]{code/msg/HistoricalRelevance.msg}
\lstinputlisting[caption=ltm/Date.msg]{code/msg/Date.msg}

\section{Interfaz ROS}\label{appendixB:interfazROS}

A continuación se presenta un ejemplo de configuración del sistema LTM, para su uso con el robot simulado. Luego se presenta el mensaje \texttt{ltm/QueryResult.msg}, utilizado por la API ROS de consulta para entregar los resultados de las búsquedas episódicas.

\lstset{style=/Style/yaml/ROS}
\lstinputlisting[caption=server.yaml,label=lst:sampleconfig]{code/server.yaml}

\lstset{style=/Style/ROS/MSG}
\lstinputlisting[caption=ltm/QueryResult.msg,label=lst:queryresult]{code/msg/QueryResult.msg}
