\chapter{Implementación}\label{chapter:implementacion}

\todowrite{Descripción del capítulo}

\todowrite{Agregar referencia al github de cada package y tutoriales respectivos.}

\section{Test suites}
% - Episodios JSON: comprensión de la estructura a utilizar
% - Episodios Smach: cubrir todos los casos de borde
% - Robot simulado: emociones y localización fakes
% - Streams Fake: videos
% - Entidades Fake: Generador de campos para mensajes

\section{Servidor LTM}
% - Mensajes
% - Genérico
% - Base de datos
% - Servicios
% - Plugins

\section{Módulos particulares a la simulación}
% - Interfaz Smach
%    - consideraciones iniciales
%    - Implementación sin impacto en librería y que minimiza trabajo del equipo
%    - consideraciones: no morir ante nada!
% - Interfaz Robot: Emociones y Posición
% - Plugin Streams
% - Entidades
% - Tutoriales

\section{Módulos particulares al robot}
% - Funcionalidades implementadas en el robot: 
%    - emociones simples.
%    - módulos disponibles (con el robot a medias)
% - Integración de ltm al sistema de UChile ROS Framework
%    - instalación y convivencia con el software
%    - visualización de datos episódicos.
% - Interfaz Smach
% - Interfaz Robot: Emociones y Posición
% - Streams
% - Entidades
