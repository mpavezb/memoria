\chapter{Implementación}\label{chapter:implementacion}

Es éste capítulo se describe la implementación del proyecto, a partir de las decisiones de diseño expuestas en el Capítulo \ref{chapter:diseno}. Primero, se presenta la estructuración del software en términos de archivos y paquetes ROS. Luego se muestran los mensajes episódicos sobre los que se construye el proyecto. En tercer lugar, se presentan módulos de software desarrollados para acelerar la implementación y validación del proyecto. Luego, se presenta la implementación del servidor LTM, junto a los módulos encargados de generar información episódica ficticia y los encargados de recolectar datos desde el robot. Finalmente, se muestra la integración del sistema en Bender.

% ==============================================================================
% ==============================================================================
% ==============================================================================
% ==============================================================================
\section{Estructura del software e instalación}
% ==============================================================================
% ==============================================================================
% ==============================================================================
% ==============================================================================

Esta sección presenta la estructuración del software implementado, en términos de sus archivos y paquetes ROS involucrados. Se describe la estructura elegida para el servidor LTM y los plugins para recolección de información episódica, junto al proceso de instalación del cada componente.

\subsection{Dependencias}

% MongoDB y warehouse_ros_mongo
% packate aún no está disponible en canal oficial para ROS kinetic, debe ser descargado desde su repositorio oficial: \url{https://github.com/ros-planning/warehouse\_ros\_mongo}


\subsection{Paquete ROS: \texttt{ltm}}
% server
% Interfaz episódica en SMACH
% sobre separación de conceptos .. 
% repositorios públicos, mantenidos en cuenta del autor del proyecto.
\url{https://github.com/mpavezb/ltm}


\subsection{Paquete ROS: \texttt{ltm\_samples}}
% Plugins episódicos para información ficticia
% contiene implementaciones que no son ...
% repositorios públicos, mantenidos en cuenta del autor del proyecto.
\url{https://github.com/mpavezb/ltm\_samples}

\todoimplementation{Separar en más packages???}


\subsection{Detalles del software}

\todowrite{Incluir figura con diagrama de funcionamiento del server - robot.}

\todowrite{Ver en que puntos se pueden agregar diagramas y figuras explicativas.}

\todowrite{Incluir tabla con Líneas de código por lenguaje... ver memoria de IAN. Usar comando: \texttt{\$ cloc ltm ltm\_samples}}


\subsection{Documentación}

.
\url{https://github.com/mpavezb/ltm} archivo \texttt{README.md}.
% información sobre instalación y tutoriales de uso del sistema.
\todoimplementation{Completar tutoriales o quitarlos del informe.}


% ==============================================================================
% ==============================================================================
% ==============================================================================
% ==============================================================================
\section{Mensajes episódicos}
% ==============================================================================
% ==============================================================================
% ==============================================================================
% ==============================================================================
% - mostrar MSG ROS implementados
% - incluir MSGs para plugins

En ésta sección se muestran los mensajes ROS implementados para manejar el concepto de episodios, los que son la base sobre la que se construye el sistema LTM y definen la estructura episódica a utilizar para la comunicación entre los clientes y el servidor en ROS.

\todounsure{Pongo código fuente de todos los mensajes definidos?.. los pongo en los anexos?.. }
\lstset{style=/Style/ROS/MSG}

% \ref{lst:episode.msg}
\lstinputlisting[caption=ltm/Episode.msg,label=lst:episode.msg]{code/msg/Episode.msg}
\lstinputlisting[caption=ltm/When.msg]{code/msg/When.msg}
\lstinputlisting[caption=ltm/Where.msg]{code/msg/Where.msg}
\lstinputlisting[caption=ltm/What.msg]{code/msg/What.msg}
\lstinputlisting[caption=ltm/Info.msg]{code/msg/Info.msg}
\lstinputlisting[caption=ltm/Relevance.msg]{code/msg/Relevance.msg}
\lstinputlisting[caption=ltm/EmotionalRelevance.msg]{code/msg/EmotionalRelevance.msg}
\lstinputlisting[caption=ltm/HistoricalRelevance.msg]{code/msg/HistoricalRelevance.msg}

% ==============================================================================
% ==============================================================================
% ==============================================================================
% ==============================================================================
\section{Herramientas para validación}
% ==============================================================================
% ==============================================================================
% ==============================================================================
% ==============================================================================
% - Episodios JSON: comprensión de la estructura a utilizar
% - Episodios Smach: cubrir todos los casos de borde
% - Robot simulado: emociones y localización fakes
% - Streams Fake: videos
% - Entidades Fake: Generador de campos para mensajes

.

% ==============================================================================
% ==============================================================================\\
% ==============================================================================
% ==============================================================================
\section{Servidor LTM}
% ==============================================================================
% ==============================================================================
% ==============================================================================
% ==============================================================================

.
% OVERVIEW
% uso de diseño explicado en la sección X
% implementación completa en C++
% sobre el manejo de la BD episódica utilizando el driver X

% MANEJO DE PLUGINS
% ==============================================================================
% ==============================================================================
\subsection{API pluginlib}
% ==============================================================================
% ==============================================================================

.
\subsubsection{Plugin: \textit{Where}}
% - API
% - convex hull

.
\subsubsection{Plugin: Emociones}
% - API

.
\subsubsection{Plugin: Streams}
% - API
% - Manejo de coleccion

.
\subsubsection{Plugin: Entidades}
% - Manejo de colecciones
% - API

.

% ==============================================================================
% ==============================================================================
\subsection{API ROS}
% ==============================================================================
% ==============================================================================
% API ROS:
% - Disponible a través de C++, Python, Java, Lisp, otros.

.

\subsubsection{Servicios implementados}
% - srv implementados
% - Servicios que provee

\lstinputlisting[caption=ltm/RegisterEpisode.srv]{code/srv/RegisterEpisode.srv}
\lstinputlisting[caption=ltm/AddEpisode.srv]{code/srv/AddEpisode.srv}
\lstinputlisting[caption=ltm/GetEpisode.srv]{code/srv/GetEpisode.srv}
\lstinputlisting[caption=ltm/UpdateTree.srv]{code/srv/UpdateTree.srv}


\subsubsection{Consultas episódicas}
% - consultas episódicas

.

\subsubsection{Configuración mediante ROS}

.

\subsubsection{Launchfiles}



% ==============================================================================
% ==============================================================================
\subsection{Trabajo futuro}
% ==============================================================================
% ==============================================================================
% Trabajo Futuro

.



% ==============================================================================
% ==============================================================================
% ==============================================================================
% ==============================================================================
\section{Módulos generadores de información ficticia}
% ==============================================================================
% ==============================================================================
% ==============================================================================
% ==============================================================================

.


% ==============================================================================
% ==============================================================================
\subsection{Plugin para recolección de \textit{Where}}
% ==============================================================================
% ==============================================================================
% WHERE
% - sólo genera al recolectar... no requiere almacenar buffer
% - configurable
% - Registro en pluginlib y configuración en el server

.


% ==============================================================================
% ==============================================================================
\subsection{Plugin para recolección de emociones}
% ==============================================================================
% ==============================================================================
% EMOTIONS
% - sólo genera al recolectar... no requiere almacenar buffer
% - configurable
% - Registro en pluginlib y configuración en el server

.



% ==============================================================================
% ==============================================================================
\subsection{Plugin para recolección de imágenes}
% ==============================================================================
% ==============================================================================
% STREAMS EN VIDEO
% - buffer
% - configurable
% - Registro en pluginlib y configuración en el server

.


% ==============================================================================
% ==============================================================================
\subsection{Plugin para recolección de entidades}
% ==============================================================================
% ==============================================================================
% ENTIDADES
% - sólo genera al recolectar... no requiere almacenar buffer
% - configurable
% - Registro en pluginlib y configuración en el server

.



% ==============================================================================
% ==============================================================================
% ==============================================================================
% ==============================================================================
\section{Módulos específicos para Bender}
% ==============================================================================
% ==============================================================================
% ==============================================================================
% ==============================================================================

.


% ==============================================================================
% ==============================================================================
\subsection{Interfaz con SMACH}
% ==============================================================================
% ==============================================================================
% INTERFAZ SMACH
% - SOBRE DISEÑO y PYTHON
% - implementación:
%   - registro de episodios y tags
%   - hooks para inicio y fin de episodios
%   - intrusividad: Implementación sin impacto en librería y que minimiza trabajo del equipo
%   - estabilidad: no morir ante nada!
% - test en todos los tipos de estado definidos?
% - ejemplo de funcionamiento
% - REVISAR MATCH CON DISEÑO ESCRITO

.



% ==============================================================================
% ==============================================================================
\subsection{Plugin para recolección de \textit{Where}}
% ==============================================================================
% ==============================================================================
% INTERFAZ WHERE
% - API del ROBOT
% - buffer y periodicidad
% - selección de posiciones
% - registro y recolección
% - Registro en pluginlib y configuración en el server
% - REVISAR MATCH CON DISEÑO ESCRITO
.
\todopaused{Escribir cuando esté implementado}


% ==============================================================================
% ==============================================================================
\subsection{Plugin para recolección de emociones}
% ==============================================================================
% ==============================================================================
% INTERFAZ EMOTIONS
% - API del ROBOT
% - buffer y periodicidad
% - selección de emocion
% - metadatos
% - registro y recolección
% - Registro en pluginlib y configuración en el server
% - REVISAR MATCH CON DISEÑO ESCRITO
% - Funcionalidades implementadas en el robot: 
%    - emociones simples.
%    - módulos disponibles (con el robot a medias)
.
\todopaused{Escribir cuando esté implementado}


% ==============================================================================
% ==============================================================================
\subsection{Plugin para recolección de imágenes}
% ==============================================================================
% ==============================================================================
% INTERFAZ IMAGES
% - API del ROBOT
.
\todopaused{Escribir cuando esté implementado}


% ==============================================================================
% ==============================================================================
\subsection{Plugins para recolección de entidades}
% ==============================================================================
% ==============================================================================
% INTERFAZ ENTITIES
% - API del ROBOT
.
\todopaused{Escribir cuando esté implementado}


% ==============================================================================
% ==============================================================================\\
% ==============================================================================
% ==============================================================================
\section{Integración del sistema en Bender}
% ==============================================================================
% ==============================================================================
% ==============================================================================
% ==============================================================================
% - instalación y convivencia con el software
% - visualización de datos episódicos.
% - Interfaz Robot: Emociones y Posición
% - Streams
% - Entidades

.
