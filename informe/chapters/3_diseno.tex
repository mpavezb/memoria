\chapter{Diseño}\label{chapter:diseno}


\section{Requerimientos}
% - Descripción verbal de requisitos y de donde se obtuvieron.
% - Listado formal de requerimientos
%   - episodicos
%   - emocionales
%   - historicos

\todo[inline]{Reescribir requerimientos de diseño y validaciones}


\section{Validaciones}
% - Sobre generación de validaciones a partir de requisitos
% - Listado formal de validaciones basadas en requerimientos
% - 

\section{Arquitectura del sistema}
% - Diseño de episodios según requisitos
%    - manejo de otros datos
%    - manejo de ubicación
%    - streams vs entidades
%    - diseño de sistema emocional
%    - manejo de datos historicos
%    - otros datos para debugging e introspección
%    - manejo de datos episodicos en hijos
%    - construcción de padres a partir de los hijos
%    - Contexto
%    - Limitantes:
%       - ubicación y frames
%       - padres mantienen con mismo frame y mapa de hijos
%       - contexto: introducción manual? TO DO

% - Diseño de Base de Datos:
%    - manejo de colecciones y mensajes
%       - límite de tamaño de datos
%    - separación de colecciones para optimizar manejo de datos y memoria.
%       - optimizar queries comunes
%       - poder bloquear/eliminar colecciones muy pesadas
%    - episodio
%    - streams
%       - funcionalidad de streams. Separación de entidades.
%       - degradación
%    - entidades
%       - cómo cumplir reglas sobre flexibilidad y perspectiva. Audit trail.
%    - otras funcionalidades de interés: bkp, migrate a otro robot, ...
%    - modificar msg de colección MD5 y poder seguir ocupando la base de datos.
%    - funcionalidades y limitantes de Mongo y de mongo_ros
%    - sistema de queries disponibles finalmente.

% - Diseño del servidor LTM según requisitos
%    - TODO: manejo de contexto
%    - uso de MongoDB
%    - Minimizar dependencias
%    - API ROS: servicios, parámetros, nodos
%    - Sistema de plugins para agregar cosas específicas a cada robot
%        - requerimientos para cada plugin
%        - uso esperado de pluginlib
%        - flujo de trabajo de cada plugin
%    - Alcances y trabajo futuro
%        - reservar ids mientras nodo esté activo.
%        - posibles funcionalidades de interés: 
%           - visualizador
%           - 

% - Plugins a implementar a modo de ejemplo, adecuados al robot bender.
%    - Smach: consideraciones
%    - Streams: dar ejemplos: img, sonido, pcl..
%    - Emociones y Localización
%    - Entidades: People, Objects, Robot, Location
