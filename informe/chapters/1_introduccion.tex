\chapter{Introducción}\label{chapter:introduction}

En este capítulo se presenta el trabajo de título a realizar: el diseño e implementación de una memoria de largo plazo episódica, semántica y emocional para el robot Bender, desarrollado en el laboratorio de robótica del Departamento de Ingeniería Eléctrica de la Universidad de Chile. Se da el contexto en el que se enmarca, la motivación para su desarrollo y los objetivos del proyecto. Se revisan brevemente conceptos sobre robótica doméstica, la memoria humana y el equipo de trabajo donde se implantará el sistema. Luego se explica el problema a tratar y la oportunidad de desarrollo. Finalmente se formalizan los objetivos del trabajo así como sus alcances.


\section{Antecedentes Generales}

A continuación, el lector encontrará una breve introducción a los temas requeridos para contextualizar este trabajo: La robótica de servicio doméstica y el equipo de trabajo donde se implantará la solución. Además, se introduce el tema de la memoria humana, requerido para entender el proyecto y su relación con la robótica. Éstos temas serán tratados en mayor profundidad más adelante, en los Capítulos~\ref{chapter:theory} y \ref{chapter:technical}.


\subsection{Robots de servicio domésticos}

La robótica de servicio es un área enfocada en asistir a los seres humanos en tareas repetitivas y comunes. Para completar una tarea, el robot requiere cierto grado de autonomía, que le permita actuar en ambientes no controlados y utilizando sus sensores para responder correctamente a los cambios del entorno~\cite{IFR}.

A grandes rasgos, los robots de servicio se categorizan en robots para el transporte, seguridad y domésticos. A su vez, los robots de servicio domésticos se caracterizan por realizar tareas de asistencia en el hogar y de compañía para humanos. Algunas de las tareas típicas que deben realizar son ayudar a ordenar, preparar comida u ofrecer bebestibles. Algunos se enfocan en el cuidado de adultos mayores, en mascotas de compañía, salud o educación.

\subsection{Liga RoboCup@Home}

RoboCup es una competencia internacional cuyo objetivo es ser un vehículo para el desarrollo de la robótica y la inteligencia artificial. Está compuesta de variadas ligas: Rescue, Soccer, Simulation, @Home, Industrial y Junior, cada una con diversas subligas orientadas a fomentar la investigación de distintos aspectos del campo. El sueño de sus organizadores, es que para mediados del siglo 21 e incorporando los desarrollos de todas las ligas, un equipo de fútbol robótico completamente autónomo sea capaz de vencer al campeón de la última copa mundial, siguiendo las reglas de la FIFA~\cite{robocup:rulebook_2018}.

Las pruebas de la liga @Home se desarrollan en escenarios que imitan ambientes domésticos reales, como un hogar o un restaurante. Las capacidades generalmente evaluadas y potenciadas son de visión computacional, navegación autónoma, manipulación de objetos y reconocimiento de voz. Sin embargo, constantemente se revisan nuevas capacidades deseables en un robot doméstico. Por otro lado, la competencia funciona como un espectáculo para público general, por lo que se priorizan pruebas y demostraciones interesantes para los espectadores.

\subsection{Equipo de trabajo: UChile Homebreakers}

El laboratorio de robótica del Departamento de Ingeniería Eléctrica de la Universidad de Chile alberga dos equipos de robótica: \textit{UChile Robotics Team}, dedicado al fútbol robótico y \textit{UChile Homebreakers Team}, enfocado en robótica de servicio. Ambos son conformados por alumnos de pregrado y postgrado de diversas especialidades, y liderados por el profesor Javier Ruiz del Solar~\cite{uchile-robotics}. UChile Homebreakers existe desde el año 2007 y actualmente cuenta con alrededor de 12 miembros. Desde sus inicios, el equipo participa desde en la competencia RoboCup@Home.

El equipo trabaja en dos plataformas humanoides de tipo doméstico, Bender y Maqui. Bender es un robot construido en el mismo laboratorio y con el objetivo de ser un mayordomo para el hogar. Maqui, desarrollado por SoftBank Robotics~\cite{softbank}, está diseñado para ser un robot de compañía. Ambos basan su desarrollo de software en ROS, un framework para facilitar la comunicación entre componentes robóticos, y con miles de usuarios alrededor del mundo~\cite{ROS:2009}. Ambos robots comparten la misma arquitectura de software y prácticamente todo su código, exceptuando los \textit{drivers} para acceder al hardware respectivo. 


\subsection{La memoria humana}

Según Eichenbaum~\cite{Eichenbaum:2008}, la memoria hace relación al almacenamiento de experiencias en el cerebro, mediante múltiples sistemas de memoria independientes y sustentados por distintas estructuras cerebrales. A grandes rasgos, la memoria se puede dividir en de corto plazo ``STM'' (Short-Term Memory) y de largo plazo ``LTM'' (Long-Term Memory). La STM maneja información muy detallada, es de poca capacidad y permite un rápido acceso, mientras que la LTM maneja información sobre experiencias y entidades, es menos detallada, de mayor capacidad y de acceso más lento.

Eichenbaum divide la LTM en una componente explícita (consciente) y una implícita (inconsciente). La parte explícita se subdivide en 2 categorías: memoria episódica y memoria semántica. La primera se encarga de almacenar episodios, para responder a las preguntas ``Qué sucedió'', ``Dónde '' y ``Cuándo''. La Memoria semántica almacena hechos y conceptos, como el lenguaje o personas y sus características. Además, la memoria explícita  almacena las conexiones entre la información semántica y los episodios relacionados. Por otro lado, la memoria implícita codifica habilidades, hábitos y preferencias.

La memoria emocional es una forma de memoria implícita que genera reacciones emocionales y sentimientos. Según los estímulos a los que se enfrente, permite modular el proceso de consolidación de la STM en LTM, modificando el nivel de relevancia de los eventos, pudiendo generar memorias muy fuertes y hábitos arraigados. Ejemplos de ésto son los flashbacks y las memorias asociadas a eventos importantes.

%Existen procesos de consolidación y deterioro de la memoria que están constantemente modificando el conocimiento episódico de la LTM. La consolidación requiere un estímulo relevante, sumado al proceso de almacenamiento, lo que genera conexiones entre la memoria episódica y la respectiva zona semántica. En caso de haber experiencias repetidas, las conexiones se fortalecen. El deterioro de la memoria es un proceso que degenera las conexiones entre ambas formas de memorias explícitas.



\section{Motivación}

La LTM es una habilidad cognitiva esencial para que un humano sienta continuidad en su vida. Al interactuar con otras personas o el ambiente les permite recordar experiencias pasadas y sus detalles. Luego, es de esperar que un robot doméstico posea una memoria que le permita potenciar sus capacidades de interacción con los humanos que ayudará~\cite{Vijayakumar2014}. Una LTM permitiría, por ejemplo, generar diálogos interesantes sobre eventos pasados o inferir aspectos del comportamiento humano, por otro lado, también permitiría la generalización de las tareas que tiene que llevar a cabo.

Durante la última década han habido algunos intentos de implementar LTM en robots domésticos, sin embargo, no existe una solución estándar y aún quedan muchas consultas sin responder~\cite{ltm_in_robocup}.

La LTM, al ser una capacidad esencial para robots domésticos, es esperable que sea agregada en el corto plazo como uno de los requisitos para RoboCup@Home. Más aún, dado el enfoque de las plataformas disponibles, Bender cómo robot mayordomo y Maqui cómo robot social, se espera que ambos posean capacidades avanzadas de interacción con los humanos, para lo que se requiere este tipo de memoria.


\subsection{Un primer acercamiento}\label{sec:primer_acercamiento}

El año 2015 se desarrolló una LTM episódica para el robot Bender, orientada a la interacción con personas y objetos~\cite{Sanchez:2015}. El trabajo consideraba métodos para almacenar, adquirir y manejar la información episódica, sumado a un proceso simple de consolidación de memoria.

Actualmente la memoria desarrollada no está operativa, ni es factible habilitarla. Se identificaron 4 causas del problema, de carácter técnico y humano:
\begin{description}
\item[Integración incompleta]
La memoria fue integrada parcialmente al software del robot, sólo con motivos de ejemplificar su funcionamiento. Por un lado, la rutina implementada requiere una pauta específica de interacción con las personas, sólo considerando la primera interacción con cada entidad, por lo que no existe actualización de los datos. Por otro lado, su desarrollo no consideró una integración con las rutinas de comportamiento utilizadas por el robot, y su estudio no indica cómo puede ser adaptado a tales rutinas de manera no intrusiva. Entonces, en la práctica no se recopila ni provee información continuamente mientras el robot está en funcionamiento.

Además, la implementación del sistema LTM no provee una API ROS para acceder a sus funcionalidades, según el estándar de los desarrollos del equipo. 

\item[Pocas entidades]
El modelo de datos está definido de forma estática y no queda claro como puede ser modificado para manejar nuevas entidades. Particularmente, la integración sólo considera 2 entidades: Persona y Objeto, para los cuales sólo se almacena información de nombre, nacionalidad e imagen. Es esperable que una memoria considere más entidades (como lugares, robots, animales u objetos) y más características para cada uno de éstas (como nombre, hobbies, trabajo o edad, para el caso específico de un humano). 

A pesar de considerar una entidad para objetos, en la práctica no se integró con los módulos relacionados que recopilan la información, por lo que realmente la memoria sólo funciona para entidades de tipo Persona.

% Prueba de concepto y evaluación subjetiva
\item[Eficiencia y escalabilidad]
El trabajo es una prueba de concepto, por lo que sólo presenta una evaluación subjetiva del sistema, a partir de un conjunto de preguntas predefinidas: el robot hace 5 consultas al usuario, para luego responder 5 consultas hechas por el usuario. Tras la interacción, el usuario responde una encuesta sobre la calidad percibida de la interacción. Por lo tanto, no existen pruebas de escalabilidad ni eficiencia de la implementación.

% Equipo no tiene incentivo para mantenerla
\item[Prioridades de UChile Homebreakers] Cada año el equipo planifica sus desarrollos de acuerdo a los requerimientos de la liga @Home, por lo que todo trabajo fuera de las áreas evaluadas no son considerados una prioridad. Luego, ya que las funcionalidades LTM no son evaluadas en la competencia, no existe un incentivo para seguir desarrollando, finalizar la integración ni mantenerla.
\end{description}

Los primeros 3 puntos son un indicador de que el diseño considerado no se ajusta a los requerimientos mínimos para un sistema LTM con memoria episódica, como los propuestos por Stachowicz~\cite{Stachowicz2012} (éstos requerimientos son extensos y se describen en detalle en la Sección~\ref{sec:ltm_exp}). Particularmente, no cumple con los siguientes requisitos: las estructuras semánticas deben poder ser definidas por el usuario, la implementación debe ser no intrusiva, eficiente en el uso de recursos del robot, y escalable para soportar el almacenamiento de miles de episodios, sin afectar los tiempos de procesamiento.

En resumen, de acuerdo a los problemas identificados, se considera que el sistema implementado es incompleto y no se ajusta a los requerimientos reales de un sistema LTM. Más importante aún, se cree el trabajo no se ajustaba a las necesidades del equipo UChile Homebreakers, pues no existió una integración que considerara las rutinas de comportamiento del robot. Por lo anterior, el sistema LTM no es utilizado y su código ha quedado obsoleto debido a la falta de mantención.

\subsection{Oportunidad}

Existe un vasto desarrollo respecto a la memoria y los procesos cognitivos, sin embargo, la investigación se concentra en campos como psicología, neurología y ciencias cognitivas. Los estudios de LTM para robots de servicio son muy acotados y no existe una solución estándar a implementar~\cite{ltm_in_robocup}. Algunos robots, como la versión comercial de Maqui, proveen aplicaciones con características de un sistema LTM, pero el código asociado no es libre, ni está basado en ROS.

El uso de LTM no está en las prioridades del equipo, sino que es algo útil para demostraciones y para potenciar la interacción humano-robot. Por ello, se considera que no basta con desarrollar un módulo capaz de recopilar información inteligentemente, sino que es necesario atacar los 4 problemas identificados del sistema LTM implementado por Sanchez et al.


\section{Objetivos del Proyecto}\label{sec:objetivos}

En esta sección se presentan los objetivos del proyecto, a modo general y en un desglose por aspectos específicos requeridos. Luego se describen los alcances y la contribución del trabajo. 

\subsection{Objetivo general}

El objetivo general es el diseño e implementación de una LTM para robots de servicio domésticos, que considere componentes episódicos, semánticos y emocionales. La LTM debe ser integrada en Bender, recopilando recuerdos constantemente y con una API acorde a los desarrollos de UChile Homebreakers. Además, se debe proveer una demostración de las funcionalidades introducidas.


\subsection{Objetivos específicos}

A continuación se presentan los objetivos específicos del proyecto, derivados a partir del objetivo general, y que representan los requerimientos clave del proyecto.
\begin{itemize}
	\item Definir un conjunto de requisitos para el diseño del sistema y las validaciones para verificar el cumplimiento de tales requisitos.
	\item Diseñar un sistema LTM para almacenamiento de memoria episódica, semántica y emocional, enfocada en robots domésticos.
	\item Diseñar el sistema LTM de manera agnóstica al robot objetivo y proveer una interfaz ROS a sus funcionalidades.
	\item Implementar el sistema LTM e integrarlo en el robot Bender.
	\item Implementar una demostración del sistema LTM en funcionamiento.
\end{itemize}

\subsection{Alcances y contribución del trabajo}

En primer lugar, se espera que este trabajo de título sirva como base para la implementación de sistemas LTM y funcionalidades derivadas en robots domésticos. Por lo tanto, el foco del proyecto es proveer un diseño robusto, basado en los objetivos propuestos, sumado a una implementación inicial que soporte el diseño. 

El proyecto provee un sistema LTM que considera la memoria emocional en su diseño. Sin embargo, Bender no dispone de módulos emocionales que puedan ser utilizados como fuente de datos para el sistema. Se considera que el desarrollo de tales módulos es un trabajo extenso y se escapa de los objetivos del proyecto. Considerando esto, el sistema LTM provee una interfaz emocional, en caso de que a futuro se implementen módulos emocionales en Bender.

Existen otras cuestiones deseables de tratar en una memoria LTM, pero que no son abordados por este proyecto. Dos ejemplos son los aspectos éticos de almacenar información sensible de usuarios, y el cómo compartir la memoria con otros robots. Estos se describen en detalle en el Capítulo~\ref{chapter:theory}.


\section{Estructura de la Memoria}
% - 1ro: introducción de conceptos generales, motivación y objetivos
% - 2do: Marco Teórico necesario para comprender el trabajo realizado.
% - 3ro: Revisión de aspectos técnicos necesarios para el trabajo. Software y Hardware.
% - 4to: Requisitos, validaciones y diseño del software.
% - 5to: Implementación del proyecto y funcionamiento
% - 6to: Resultados de las validaciones y análisis
% - 7to: Conclusiones y trabajo futuro.

En este primer capítulo se hizo una descripción breve de los conceptos necesarios para introducir el proyecto, su motivación y sus objetivos. El Capítulo~\ref{chapter:theory} hace una revisión de los aspectos teóricos necesarios para comprender el trabajo realizado, y el Capítulo~\ref{chapter:technical} describe los componentes de software y hardware requeridos para el proyecto. Luego, el Capítulo~\ref{chapter:diseno} presenta los requisitos del proyecto, las validaciones propuestas y el diseño del sistema LTM, mientras que en el Capítulo~\ref{chapter:implementacion} se describe su implementación y funcionamiento. En el Capítulo~\ref{chapter:results} se muestran los resultados de las validaciones y su análisis. Finalmente, se presentan las conclusiones y el trabajo futuro propuesto para el proyecto.

%\section{Resumen}



