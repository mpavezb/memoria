\chapter{Introducción}\label{chapter:introduction}

\todo[inline]{INTRO: Revisión y corrección de lo migrado desde preinforme.}
\todo[inline]{INTRO: Corrección de referencias y cosas obsoletas.}
\todo[inline]{Ver que tanto se puede modificar el formato del template.. por ejemplo, para cambiar como se muestran los capítulos: \url{https://es.sharelatex.com/learn/Sections\_and\_chapters}}

En este capítulo se presenta el trabajo de título a realizar. Se da el contexto en el que se enmarca, la motivación para su desarrollo y los objetivos del proyecto. Se revisan los conceptos de robótica doméstica, la memoria humana y el equipo de trabajo donde se implantará el software. Luego se detallan los problemas actuales y la oportunidad de desarrollo. Finalmente se formalizan los objetivos del trabajo así como sus alcances.


\section{Antecedentes Generales}

A continuación, el lector encontrará una breve introducción a los temas requeridos para contextualizar este trabajo: La robótica de servicio doméstica y el equipo de trabajo donde se implantará la solución. Además, se introduce el tema de la memoria humana, requerido para entender la propuesta y su relación con la robótica. Todos estos temas serán tratados en mayor profundidad más adelante, en los Capítulos \ref{chapter:memory} y \ref{chapter:technical}.


\subsection{Robots de servicio domésticos}
% - 
% - 
% - 

La robótica de servicio es un área enfocada en asistir a los seres humanos en tareas repetitivas y comunes, como la recolección de basura. Para completar una tarea, el robot requiere cierto grado de autonomía, que le permita actuar en ambientes no controlados y utilizando sus sensores para responder correctamente a los cambios del entorno.

A grandes rasgos, los robots de servicio se categorizan en robots para el transporte, seguridad y domésticos. Los robots de servicio domésticos se caracterizan por realizar tareas de asistencia en el hogar y de compañía para humanos. Algunas de las tareas típicas que deben realizar son ayudar a ordenar, preparar comida u ofrecer bebestibles. Algunos se enfocan en el cuidado de adultos mayores, en mascotas de compañía, salud o educación.


\subsection{Equipo de trabajo: UChile Homebreakers}
% - 
% - 
% - 
El laboratorio de robótica del Departamento de Ingeniería Eléctrica de la Universidad de Chile alberga dos equipos de robótica: \textit{UChile Robotics Team}, dedicado al fútbol robótico y \textit{UChile Homebreakers Team}, enfocado en robótica de servicio. Ambos son conformados por alumnos de pregrado y postgrado de diversas especialidades, y liderados por el profesor Javier Ruiz del Solar \cite{uchile-robotics}.

UChile Homebreakers existe desde el año 2007 y actualmente cuenta con 15 estudiantes. Todo su desarrollo de software está basado en ROS, un framework para el desarrollo de plataformas robóticas y con miles de usuarios alrededor del mundo \cite{ROS:2009}.

El equipo trabaja en dos plataformas humanoides de tipo doméstico, Bender y Pepper. Bender es un robot construido en el mismo laboratorio y con el objetivo de ser un mayordomo para el hogar. Pepper, desarrollado por SoftBank Robotics \cite{softbank}, está diseñado para ser un robot de compañía. Ambos comparten la misma arquitectura de software y prácticamente todo su código, exceptuando los drivers para acceder al hardware respectivo.


\unsure[inline]{Es necesario hablar de la robocup??}

%\subsubsection{RoboCup @Home League}
%
%La RoboCup es una competencia internacional cuyo objetivo es ser un vehículo para el desarrollo de la robótica y la inteligencia artificial. Está compuesta de variadas ligas: Rescue, Soccer, Simulation, @Home, Industrial y Junior, cada una con diversas subligas orientadas a fomentar la investigación de distintos aspectos del campo. Su sueño es que para mediados del siglo 21, un equipo de fútbol robótico completamente autónomo sea capaz de vencer al campión de la última copa mundial y siguiendo las reglas de la FIFA \cite{robocup:rulebook_2017}.
%
%UChile Homebreakers participa desde el año 2007 en la categoría @Home. Las pruebas de la liga se desarrollan en escenarios que imitan ambientes reales, como un hogar o un restaurante. 
%%Además, la competencia funciona como un espectáculo para público general, por lo que se priorizan pruebas y demostraciones interesantes para los espectadores.
%% 
%Las capacidades generalmente evaluadas y potenciadas en @Home son de Visión Computacional, Navegación autónoma, Manipulación de objetos y Reconocimiento de Voz. Cada año el equipo planifica sus desarrollos de acuerdo a los requerimientos de la competencia, por lo que trabajos fuera de las áreas mencionadas no son considerados una prioridad.


\subsection{La memoria humana}
% - 
% - 
% - 
Según Eichenbaum \cite{Eichenbaum:2008}, la memoria hace relación al almacenamiento de experiencias en el cerebro. Se pueden distinguir múltiples sistemas de memoria independientes y sustentados por distintas estructuras cerebrales. A grandes rasgos, la memoria se puede dividir en de corto plazo STM (Short-Term Memory) y de largo plazo LTM (Long-Term Memory). La STM maneja información muy detallada, es de poca capacidad y permite un rápido acceso, mientras que la LTM maneja mucha información sobre experiencias y entidades, es menos detallada y de acceso más lento.

Eichenbaum divide la LTM en una componente explícita (consciente) y una implícita (inconsciente). La primera almacena datos episódicos, pudiendo responder las preguntas ``Qué'', ``Dónde'' y ``Cuándo'', datos semánticos, que modelan hechos y conceptos como el lenguaje o personas, y también, las conexiones entre ambas submemorias. La memoria implícita codifica habilidades, hábitos y preferencias.

Existen procesos de consolidación y deterioro de la memoria que están constantemente en funcionamiento. La consolidación requiere un estímulo relevante, sumado al proceso de almacenamiento, lo que genera conexiones entre la memoria episódica y la respectiva zona semántica. En caso de haber experiencias repetidas, las conexiones se fortalecen. El deterioro de la memoria es un proceso que degenera las conexiones entre ambas formas de memorias explícitas.

La memoria emocional es una forma de memoria implícita que genera reacciones emocionales y sentimientos. Según los estímulos a los que se enfrente, permite modular el proceso de consolidación de la STM en LTM, modificando el nivel de relevancia de los eventos, pudiendo generar memorias muy fuertes y hábitos arraigados. Ejemplos de esto son los flashbacks y las memorias asociadas a eventos importantes.




\section{Motivación}

\todo[inline]{Intro: Importancia de la memoria a largo plazo}

La memoria es una habilidad cognitiva crucial para los humanos. Al interactuar con otras personas o el ambiente les permite recordar experiencias pasadas y sus detalles. Luego, es de esperar que un robot de servicio posea una memoria que le permita potenciar sus capacidades de interacción con los humanos que ayudará \cite{Vijayakumar2014}. Una LTM permitiría, por ejemplo, generar diálogos interesantes sobre eventos pasados o cosas que el robot puede inferir del comportamiento humano, por otro lado, también permitiría la generalización de las tareas que tiene que llevar a cabo.

Particularmente, dado el enfoque de las plataformas disponibles, Bender cómo robot mayordomo y Pepper cómo robot social, se espera que ambos posean capacidades avanzadas de interacción con los humanos, para lo que se requiere una LTM.


\subsection{Problema}
% - Implementación anterior de memoria a largo plazo
%    - No integrada correctamente
%    - Muy limitada
%    - Requiere de mucha intervención del equipo!. No mantenida
% - No existen software alternativos para resolver este problema.

El año 2015 se desarrolló una LTM episódica para el robot Bender, orientada a la interacción con personas y objetos \cite{Sanchez:2015}. El trabajo consideraba métodos para almacenar, adquirir y manejar la información episódica, sumado a un proceso simple de consolidación de memoria.

Actualmente la memoria desarrollada no está operativa, ni es factible habilitarla. A continuación se listan los aspectos que se consideran causas del problema desde un punto de vista técnico y humano:
\begin{itemize}
	\item No se integró adecuadamente al software del robot, no se recopila ni provee información continuamente mientras el robot está en funcionamiento.
	\item La memoria no provee una API que siga el estándar de los desarrollos del equipo, por lo que no se usa ni es mantenida.
	\item RoboCup@Home no considera el uso de LTM en sus competencias, por lo que el equipo no tiene un incentivo real para seguir desarrollando o mantener la memoria. Esto además ha provocado que el código quede obsoleto.
\end{itemize}

Por otro lado, suponiendo que lo anterior estuviese solucionado, aún existen los siguientes problemas:
\begin{itemize}
	\item Sólo considera 2 modelos semánticos: Persona y Objeto, para los cuales sólo se almacena información de nombre, nacionalidad e imagen.
	\item A pesar de considerar un modelo para objetos, no se integró con los módulos relacionados que recopilan la información, por lo que realmente la memoria sólo funciona para entidades de tipo Persona.
	\item Es esperable que una memoria considere más modelos (como adultos, niños, animales u objetos) y más características para cada uno de los modelos (como nombre, hobbies, trabajo o edad, para el caso específico de un humano).
	\item La consolidación de memoria STM a LTM sólo considera la primera interacción con cada entidad, por lo que no existe actualización de los datos.
	\item Hay una restricción en los modelos y características a almacenar, respecto a la información que el robot es realmente capaz de obtener.
\end{itemize}



\subsection{Oportunidad}
% - Diseño basado en requerimientos episódicos (cumplirlos todos)
% - Servidor genérico que no fuerce implementaciones y no limitado a sólo 1 robot
% - Arquitectura no invasiva y mantenible
% - Capacidad de integración a futuro en módulos de inferencia

Existe un vasto desarrollo respecto a la memoria y los procesos cognitivos, sin embargo, la investigación se concentra en campos como psicología, neurología y ciencias cognitivas\todo{cita aqui!}. Los estudios de LTM para robots de servicio son muy acotados y no existe una solución estándar a implementar. Algunos robots, como la versión comercial de Pepper, utilizan LTM, pero el código asociado no es libre, ni está basado en ROS.

El uso de LTM no está en las prioridades ``RoboCup'' del equipo, sino que es algo útil para demostraciones y para potenciar la interacción humano-robot. Por ello, se considera que no basta con desarrollar un módulo capaz de recopilar información inteligentemente, sino que además se requiere una integración con las capacidades de diálogo o de inferencia de información, para finalmente proveer una demostración de estas habilidades.

Así, esta es una oportunidad para diseñar una LTM para robots de servicio, que considere aspectos como: 
\begin{itemize}
	\item Memoria episódica y semántica adecuada a tareas generales de robots de servicio.
	\item Metodología para consolidación de STM en LTM.
	\item Servicio para recopilación continua de información.
	\item Implementación basada en el framework ROS, siguiendo la línea de los desarrollos en UChile Homebreakers.
	%\item Capacidad de generar respaldos de la memoria y recuperación de éstos. 
	\item Memoria emocional que permita dar relevancia a los eventos.
	\item Inferencia de información a partir de datos de la memoria. Por ejemplo: ``Juan suele desayunar a las 9 am'', ``El control de la TV suele estar en el sofá'', etc.
	%\item Integración con el diálogo que realiza el robot.\unsure{quitarlo?}
\end{itemize}

Tanto la memoria emocional como la inferencia de información se consideran requisitos deseables, por lo que están fuera del \textit{core} del proyecto.

Más adelante se detallan cada uno de los aspectos anteriores. En el Capítulo \ref{chapter:memory} se hace una revisión de cada punto, respecto al estado del arte. Mientras que en el Capítulo \ref{chapter:metodologia} se propone una alternativa de solución a cada uno de ellos.

%También se considera que es la oportunidad de promover la inclusión de desafios basados en LTM en la liga @Home, a partir de los resultados de este trabajo. Así, el desarrollo de LTMs y capacidades asociadas dejaría de ser postergado y pasaría a ser una prioridad para los equipos de la competencia.



\section{Objetivos del Proyecto}

\subsection{Objetivo general}
% - 
% - 
% - 
El objetivo general es el diseño de una LTM para robots de servicio domésticos, que considere componentes episódicos y semánticos. La LTM debe ser integrada en Bender, recopilando recuerdos constantemente y con una API acorde a los desarrollos de UChile Homebreakers. Además, se debe proveer una demostración de las funcionalidades introducidas.


\subsection{Objetivos específicos}
% - 
% - 
% - 
A continuación se presentan los objetivos específicos del trabajo, a modo de desglose del objetivo general en tareas más acotadas.

\begin{itemize}
	\item Definir consultas para validación del sistema.
	\item Diseñar el proceso de consolidación de recuerdos.
	\item Diseñar la arquitectura del sistema.
	\item Implementar la LTM y su API.
	\item Implementar servicio que recopile recuerdos constantemente.
	\item Implementar la demostración.
\end{itemize}


\subsection{Alcances y contribución del trabajo}
% - 
% - 
% - 
Se espera que este trabajo de título sirva como base para el desarrollo de funcionalidades más avanzadas, basadas en memorias LTM en el futuro. Por lo tanto, el foco del trabajo será el diseño de la LTM y la implementación del software que soporte el diseño. La demostración sólo será utilizada como medio de validación del trabajo. Entonces, la principal contribución del trabajo es el diseño de la LTM para Bender.

Tanto la memoria emocional, como la inferencia de información se consideran objetivos secundarios, que no son del \textit{core} del proyecto. Sin embargo, dado que son conceptos casi imprescindibles para una LTM, deben ser considerados en el diseño, a pesar de que no sean implementados.

En términos del desafío del trabajo, y tras una revisión del estado del arte y el software disponible (Capítulos \ref{chapter:memory} y \ref{chapter:technical}), se cree que la mayor parte del esfuerzo estará en el diseño de una LTM Episódica compatible con URF y KnowRob. Tanto URF como KnowRob  son revisados en el Capítulo \ref{chapter:technical}. URF hace referencia a todo el software actualmente funcionando en el robot Bender. Se propone el uso del software KnowRob como base para la implementación de la memoria, pues está diseñado para proveer S-LTM, P-LTM e inferencias sobre ambas memorias.


\section{Estructura de la Memoria}
% - 
% - 
% - 
\todo[inline]{Sobre la estructura de la memoria}