\documentclass[12pt,letterpaper,spanish]{article}

% formato, lenguaje y encoding
\usepackage[utf8]{inputenc}
\usepackage[spanish, es-nolists, es-lcroman]{babel}
\usepackage[top=3cm, left=3cm, bottom=2cm, right=2cm, paper=letterpaper]{geometry}


% Links y numeracion del PDF
\usepackage[pdfpagelabels]{hyperref}

% Codigo Fuente
\usepackage{listings}

% varios
\usepackage{amsmath, amssymb, amsthm, graphicx}


% = = = = = = = = = = = = = = = = = = = = = = =
% TODO NOTES
% = = = = = = = = = = = = = = = = = = = = = = =
\usepackage[pdftex,dvipsnames]{xcolor}  % Coloured text etc.
\usepackage{xargs} % Use more than one optional parameter in a new commands
\usepackage[colorinlistoftodos,prependcaption,textsize=tiny]{todonotes}
\newcommandx{\unsure}[2][1=]{\todo[linecolor=red,backgroundcolor=red!25,bordercolor=red,#1]{#2}}
\newcommandx{\change}[2][1=]{\todo[linecolor=blue,backgroundcolor=blue!25,bordercolor=blue,#1]{#2}}
\newcommandx{\info}[2][1=]{\todo[linecolor=OliveGreen,backgroundcolor=OliveGreen!25,bordercolor=OliveGreen,#1]{#2}}
\newcommandx{\improvement}[2][1=]{\todo[linecolor=Plum,backgroundcolor=Plum!25,bordercolor=Plum,#1]{#2}}
\newcommandx{\thiswillnotshow}[2][1=]{\todo[disable,#1]{#2}}
% = = = = = = = = = = = = = = = = = = = = = = =

\begin{document}


%----------------------------------------------------------------------------------------
\begin{titlepage}

\newcommand{\HRule}{\rule{\linewidth}{0.5mm}}

\center % Center everything on the page
 
\ \\[-1cm]
\includegraphics[height=3cm]{escudoU2014.pdf}\\[0.5cm]
\textsc{\LARGE Universidad de Chile}\\[0cm]
\textsc{\large Departamento de Ingenier\'ia El\'ectrica}\\
\textsc{\large Departamento de Ciencias de la Computaci\'on}\\[3cm]

\HRule \\[0.4cm]
{ \huge \bfseries Dise\~no e Implementaci\'on de Memoria de Largo Plazo para Robots de Servicio}\\[0.1cm] % Title of your document
\HRule \\[0.5cm]

\textsc{\large Propuesta de Memoria para optar al T\'itulo de Ingeniero Civil El\'ectrico e Ingeniero Civil en Computaci\'on.}\\[1cm]

\begin{minipage}{0.4\textwidth}
\begin{flushleft} \large
\emph{Autor:}\\
Mat\'ias \textsc{Pavez}\\
\texttt{\normalsize matias.pavez@ing.uchile.cl} \\
\texttt{\normalsize +569 9888 9358}
\end{flushleft}
\end{minipage}
~
\begin{minipage}{0.4\textwidth}
\begin{flushright} \large
\emph{Profesor Gu\'ia:} (DIE) \\
Javier \textsc{Ruiz del Solar}\\
\texttt{\normalsize jruizd@ing.uchile.cl}
\end{flushright}
\end{minipage}\\[1cm]
\begin{minipage}{0.4\textwidth}
\begin{flushleft}\end{flushleft}
\end{minipage}
~
\begin{minipage}{0.4\textwidth}
\begin{flushright} \large
\emph{Co-Gu\'ia:} (DCC)\\
Jocelyn \textsc{Simmond}\\
\texttt{\normalsize jsimmond@dcc.uchile.cl}
\end{flushright}
\end{minipage}\\[2.5cm]
\improvement[inline]{FIRMAS FIRMAS FIRMAS FIRMAS FIRMAS FIRMAS FIRMAS FIRMAS FIRMAS FIRMAS FIRMAS FIRMAS FIRMAS FIRMAS FIRMAS}
{\large Santiago de Chile}\\
{\large Abril, 2017}
\vfill
 
\end{titlepage}
%----------------------------------------------------------------------------------------

%\improvement[inline]{REVISAR REDACCI\'ON, P\'ARRAFOS CORRECTOS}

\section{Contexto}

%\improvement[inline]{los conceptos necesarios para introducir al lector en el tema y permitirle entender el resto de la presentacion.}

\subsection{Robots de Servicio}

La rob\'otica de servicio es un \'area enfocada en asistir a los seres humanos en tareas repetitivas y comunes, como los deberes del hogar. Formalmente, se define un \textit{robot de servicio} como un robot ``que realiza tareas \'utiles para humanos o equipamiento, excluyendo aplicaciones de automatizaci\'on industrial''\cite{IFR}. Luego, el robot requiere cierto grado de autonom\'ia, que es la habilidad de actuar a partir del estado actual, usando lo que observa del ambiente y sin intervenci\'on humana. As\'i, un robot de servicio debe trabajar en ambientes no controlados y con la autonom\'ia suficiente que le permita llevar a cabo su cometido.\\

Un caso de uso t\'ipico es la asistencia en las tareas del hogar, donde se espera que un robot pueda ayudar a ordenar, preparar comida u ofrecer bebestibles. Otros casos de uso consideran el cuidado de adultos mayores, robots para compa\~nia en el hogar, mascotas robots, salud, educaci\'on o para la recolecci\'on de basura. Particularmente, la compa\~nia SoftBank Robotics es pionera en ofrecer a Pepper como el primer robot humanoide ya adoptado en hogares de Jap\'on, as\'i como robot de bienvenida en hoteles y tiendas\cite{softbank}.\\


\subsection{Laboratorio de Rob\'otica y Equipo de Trabajo}

En laboratorio de rob\'otica del Departamento de Ingenier\'ia El\'ectrica de la Universidad de Chile, alberga dos equipos de rob\'otica: \textit{UChile Robotics Team}, dedicado al fútbol rob\'otico y \textit{UChile Homebreakers Team}, enfocado en rob\'otica de servicio. Ambos son conformados por alumnos de pregrado y postgrado de diversas especialidades, y liderados por el profesor Javier Ruiz del Solar\cite{uchile-robotics}.\\

UChile Homebreakers existe desde el a\~no 2007 y actualmente cuenta con 15 estudiantes. Todo su desarrollo de software est\'a basado en ROS, un framework para el desarrollo de plataformas rob\'oticas y con miles de usuarios alrededor del mundo\cite{ROS:2009}.\\

El equipo trabaja en dos plataformas humanoides, Bender y Pepper. Bender es un robot construido en el laboratorio y con el objetivo de ser un mayordomo en el hogar. Pepper, desarrollado por SoftBank Robotics, est\'a dise\~nado para ser un robot de compa\~nia.\\


\subsection{RoboCup @Home League}

La RoboCup es una competencia internacional cuyo objetivo es ser un veh\'iculo para el desarrollo de la rob\'otica y la inteligencia artificial. Est\'a compuesta de variadas ligas: Rescue, Soccer, Simulation, @Home, Industrial y Junior, cada una con diversas subligas orientadas a fomentar la investigaci\'on de distintos aspectos del campo. Su sue\~no es que para mediados del siglo 21, un equipo de futbol rob\'otico completamente aut\'onomo, sea capaz de vencer al campi\'on de la última copa mundial y siguiendo las reglas de la FIFA\cite{robocup:rulebook_2017}.\\ 

UChile Homebreakers participa desde el a\~no 2007 en la categor\'ia @Home. Las pruebas de la liga se desarrollan en escenarios que imitan ambientes reales, como un hogar o un restaurante. Las capacidades evaluadas y potenciadas son de Vision Computacional, Navegaci\'on aut\'onoma, Manipulaci\'on de objetos y Reconocimiento de V\'oz. Cada a\~no el equipo planifica sus desarrollos de acuerdo a los requerimientos de la competencia, por lo que trabajos fuera de las \'areas mencionadas no son considerados una prioridad.\\

%\subsection{Interacci\'on Humano Robot}
%
%La Interacci\'on Humano-Robot (HRI, por sus siglas en ingl\'es), ...
%
%Sobre HRI y la robocup
%
%Bender y Pepper... mostrar emociones, Speech Recognition, conversar con gente... objetivos..
%% Luego, ambos deben ser robots con alto desarrollo del \'area de Human-Robot interaction. \\

\section{Motivaci\'on}

\todo[inline]{por que el tema es interesante como memoria}

\todo[inline]{por que es suficientemente complejo}

\todo[inline]{que problemas a resolver}

\todo[inline]{que interes academico o profesional tiene}

\subsection{Problema}

Sobre Loreto  \cite{Sanchez:2015} y Victoria.
memoria dise\~nada e implementada parcialmente.. 
no integrada en el robot, no se usa, desactualizacion
no provee una API ROS bender estandar (skills).. no hay incentivo para ocuparla.
sólo tiene 2 modelos semánticos: personas y objetos .. otros modelos de interes?
falta la integraci\'on con objetos
informacion acotada para modelo de personas ... otros datos de interes??

equipo:
no hay incentivo robocup para mantener el desarrollo de la memoria. (sólo queda hacerla parte del SO y que funcione continuamente... se espera que esto cambie)
demos-significancia de la robotica de servicio (capacidad de HRI) .. interesante desarrollar capacidades de HRI.

hay limitacion respecto a los datos que realmente es posible obtener (oportunidad, trabajo futuro-). ejemplos de datos que actuallmente se tienen.



%Hasta la actualidad, . Sin embargo, casi no ha considerado el tema de los recuerdos Sobre requerimientos de memoria bajos, pero inter\'es en potenciar esta \'area.\\
%
%... mi trabajo pretende romper el esquema actual y dar un avance en la implementaci\'on  de memoria e inferencia.\change{esto es m\'as como objetivos o oportunidad}
%area muy poco exploirada \improvement{referencias}
%pretende proveer el c\'odigo fuente de la forma estandar que se utiliza en ROS
%pretende mostrar la importancia de este tema en HRI e incentivar la consideraci\'on de est en la competencia...\improvement{hay otras competencias que enfaticen esto??}




\subsection{Oportunidad}

Existe poco desarrollo al respecto y no hay soluciones evidentes ni estandar.. tampoco existe codigo libre para esto. \\

Diseño de memoria a largo plazo para robots de servicio
- base com\'un + componentes custom
- que otros modelos especializados son de inter\'es,, casas, ni\~nos mascotas, autos..
- quefeatures son relevantes para ellos
- sobre los datos: almacenamiento, formato, compresi\'on
- datos relevantes para tareas futuras, cuales, planes, mapas..
consolidacion de STM a LTM
memoria emocional y estimulos relevantes
consolidacion vs. deterioro de la informacion
actualizacion y envejecimiento de las entidades.


Implementacion
. estandar ROS. skills
integracion en robots.. funcional, recopilacion continua..servicios en background
. interaccion con memoria a corto plazo ... es algo difuso aun.. hay 4 fuentes de info: ed, worldmodel, context, nodos ROS

Implementacion Propuestos
- inferir info .. suele desayunar a las 9am
- visualizacion de los datos
- migracion y memoria compartida con otros robots
- memoria emocional; personalidad y estado de animo del robot
- complementar HRI mediante info WEB
- almacenaje, merge y recuperacion de bakups.

Validacion
- Definir preguntas a responder
- queries episodicas: que hiciste ayer, como?, que paso hace 1 mes?
- queries semanticas: que ha cambiado en el entorno desde tu ultima visita?, hace 1 a\~no?. Describe a Juan.. Hablame de Pedro...
- definir otras validaciones intrinsecas
- definir otras validaciones dependientes de las implementaciones propuestas.

como validar:
- v0. funcion(skills) que retorme informacion solicitada.
- v1. mostrar info solicitada en visualizador (de seguir ese camino)
- v2. uso de API externa para analisis de frases e intenciones
- v3. uso de chatbot para demo de conversacion.


La memoria no s\'olo se relaciona a HRI, sino que le puede servir al robot para inferir datos.... mejorar la realizaci\'on de actividades, dar una personalidad.. ?? NADA de eso es HRI??


\section{Posibles Soluciones}

\subsection{Dise\~no de memoria}

\subsubsection{Approach basado en la Memoria Humana...}

Sobre Felix ... \unsure{esto no se puede citar de alguna forma??.. c\'omo lo referencio?}\\

Sobre la memoria \cite{Eichenbaum:2008}\\
... memoria de corto plazo STM> iconica u de trabajo.. info detallada sobre eventos recientes.\\

.. memoria de largo plazo (LTM).\\
- episodica: responde a preguntas qhat, where and when.. dando links a areas de la memoria  semantica.\\
- semantica: info estructurada en modelos especializados: Personas, Objetos, Lenguaje, Musica, ...\\

... memoria emocional (implicita)\\
- modula la consolidacion de la STM en LTM\\
- permite dar mas importancia a los eventos... mas alla del replay.\\
- ejemplos de esto.\\

... plasticidad: procesos de consolidacion y deterioro de la memoria.\\
- consolidacionL: se requiere estimulo + sinapsis. siempre fortalecen los links a partir del replay o info nueva (interesante)\\
- todos los links no utilizados estan constantemente debilitandose. no se borra la memoria, sino que se debilitan los links.\\

En que se traduce esto?... lo de loreto. Entidades de la Base se datos.\\


\subsubsection{Approach 2 ..}


\subsection{Diseño de bla...}

El resto...

\section{Objetivos}

\subsection{Alcances y Objetivo General}

Dise\~no e implementaci\'on de memoria de largo plazo para robots de servicio.\change{Explicar m\'as detalaldo.. con otras palabras.}
\improvement[inline]{unas pocas l\'ineas describiendo a grandes rasgos cual es el objetivo que se persigue con la memoria. Evitar que el objetivo general sea simplemente una repeticion del t\'itulo.}

\subsection{Objetivos Espec\'ificos}

\improvement[inline]{na serie de puntos mas detallados que desglosen el objetivo general en una serie de objetivos menores verificables.}

- Evaluar requerimientos de ribits de servicio. Bender/Pepper.
- Definir preguntas de interes a validar
- modelos features de interes
- datos a recuperar y almacenamiento.
- consolidacion y deterioro, memoria emocional
- implementacion de memoria. Recopilacion continue y API para desarrolladores
- validaciones
- otros dependen de los componentes escogidos.


\section{Idea General de la Soluci\'on}

\todo[inline]{TODO, por definir}


\section{Metodolog\'ia}

\improvement[inline]{de que forma se planea abordar la memoria, es decir que serie de pasos se seguiran para cumplir los objetivos. Si se desea se puede incluir un cronograma de trabajo.}

- revisar diapos de Ing. de Software. Sobre metodologias de trabajo y ciclo de software.


\begin{itemize}
\item Revisi\'on Bibliogr\'afica
\item 
\end{itemize}


\bibliographystyle{IEEEtran}
\bibliography{IEEEabrv,bibliography.bib}

\end{document}