\documentclass[12pt,letterpaper,spanish]{article}

% formato, lenguaje y encoding
\usepackage[utf8]{inputenc}
\usepackage[spanish, es-nolists, es-lcroman]{babel}
\usepackage[top=3cm, left=3cm, bottom=3cm, right=2cm, paper=letterpaper]{geometry}


% Links y numeracion del PDF
\usepackage[pdfpagelabels]{hyperref}

% Codigo Fuente
\usepackage{listings}

% varios
\usepackage{amsmath, amssymb, amsthm, graphicx}


% = = = = = = = = = = = = = = = = = = = = = = =
% TODO NOTES
% = = = = = = = = = = = = = = = = = = = = = = =
\usepackage[pdftex,dvipsnames]{xcolor}  % Coloured text etc.
\usepackage{xargs} % Use more than one optional parameter in a new commands
\usepackage[colorinlistoftodos,prependcaption,textsize=tiny]{todonotes}
\newcommandx{\unsure}[2][1=]{\todo[linecolor=red,backgroundcolor=red!25,bordercolor=red,#1]{#2}}
\newcommandx{\change}[2][1=]{\todo[linecolor=blue,backgroundcolor=blue!25,bordercolor=blue,#1]{#2}}
\newcommandx{\info}[2][1=]{\todo[linecolor=OliveGreen,backgroundcolor=OliveGreen!25,bordercolor=OliveGreen,#1]{#2}}
\newcommandx{\improvement}[2][1=]{\todo[linecolor=Plum,backgroundcolor=Plum!25,bordercolor=Plum,#1]{#2}}
\newcommandx{\thiswillnotshow}[2][1=]{\todo[disable,#1]{#2}}
% = = = = = = = = = = = = = = = = = = = = = = =

\begin{document}


%----------------------------------------------------------------------------------------
\begin{titlepage}

\newcommand{\HRule}{\rule{\linewidth}{0.5mm}}

\center % Center everything on the page
 
\ \\[-1cm]
\includegraphics[height=3cm]{escudoU2014.pdf}\\[0.5cm]
\textsc{\LARGE Universidad de Chile}\\[0cm]
\textsc{\large Departamento de Ingenier\'ia El\'ectrica}\\
\textsc{\large Departamento de Ciencias de la Computaci\'on}\\[3cm]

\HRule \\[0.4cm]
{ \huge \bfseries Dise\~no e Implementaci\'on de Memoria de Largo Plazo para Robots de Servicio}\\[0.1cm] % Title of your document
\HRule \\[0.5cm]

\textsc{\large Propuesta de Memoria para optar al T\'itulo de Ingeniero Civil El\'ectrico e Ingeniero Civil en Computaci\'on.}\\[1cm]

\begin{minipage}{0.4\textwidth}
\begin{flushleft} \large
\emph{Autor:}\\
Mat\'ias \textsc{Pavez}\\
\texttt{\normalsize matias.pavez@ing.uchile.cl} \\
\texttt{\normalsize +569 9888 9358}
\end{flushleft}
\end{minipage}
~
\begin{minipage}{0.4\textwidth}
\begin{flushright} \large
\emph{Profesor Gu\'ia:} (DIE) \\
Javier \textsc{Ruiz del Solar}\\
\texttt{\normalsize jruizd@ing.uchile.cl}
\end{flushright}
\end{minipage}\\[1cm]
\begin{minipage}{0.4\textwidth}
\begin{flushleft}\end{flushleft}
\end{minipage}
~
\begin{minipage}{0.4\textwidth}
\begin{flushright} \large
\emph{Co-Gu\'ia:} (DCC)\\
Jocelyn \textsc{Simmond}\\
\texttt{\normalsize jsimmond@dcc.uchile.cl}
\end{flushright}
\end{minipage}\\[2cm]
{\large Santiago de Chile}\\
{\large Abril, 2017}
\vfill
 
\end{titlepage}
%----------------------------------------------------------------------------------------

%\improvement[inline]{REVISAR REDACCI\'ON, P\'ARRAFOS CORRECTOS}

\section{Contexto}

%\improvement[inline]{los conceptos necesarios para introducir al lector en el tema y permitirle entender el resto de la presentacion.}

\subsection{Robots de Servicio}

La rob\'otica de servicio es un \'area enfocada en asistir a los seres humanos en tareas repetitivas y comunes, como los deberes del hogar. Formalmente, se define un \textit{robot de servicio} como un robot ``que realiza tareas \'utiles para humanos o equipamiento, excluyendo aplicaciones de automatizaci\'on industrial''\cite{IFR}. Luego, el robot requiere cierto grado de autonom\'ia, que es la habilidad de actuar a partir del estado actual, usando lo que observa del ambiente y sin intervenci\'on humana. As\'i, un robot de servicio debe trabajar en ambientes no controlados y con la autonom\'ia suficiente que le permita llevar a cabo su cometido.\\

Un caso de uso t\'ipico es la asistencia en las tareas del hogar, donde se espera que un robot pueda ayudar a ordenar, preparar comida u ofrecer bebestibles. Otros casos de uso consideran el cuidado de adultos mayores, robots para compa\~nia en el hogar, mascotas robots, salud, educaci\'on o para la recolecci\'on de basura. Particularmente, la compa\~nia SoftBank Robotics es pionera en ofrecer a Pepper como el primer robot humanoide ya adoptado en hogares de Jap\'on, as\'i como robot de bienvenida en hoteles y tiendas\cite{softbank}.\\


\subsection{Laboratorio de Rob\'otica y Equipo de Trabajo}

En laboratorio de rob\'otica del Departamento de Ingenier\'ia El\'ectrica de la Universidad de Chile, alberga dos equipos de rob\'otica: \textit{UChile Robotics Team}, dedicado al fútbol rob\'otico y \textit{UChile Homebreakers Team}, enfocado en rob\'otica de servicio. Ambos son conformados por alumnos de pregrado y postgrado de diversas especialidades, y liderados por el profesor Javier Ruiz del Solar\cite{uchile-robotics}.\\

UChile Homebreakers existe desde el a\~no 2007 y actualmente cuenta con 15 estudiantes. Todo su desarrollo de software est\'a basado en ROS, un framework para el desarrollo de plataformas rob\'oticas y con miles de usuarios alrededor del mundo\cite{ROS:2009}.\\

El equipo trabaja en dos plataformas humanoides, Bender y Pepper. Bender es un robot construido en el laboratorio y con el objetivo de ser un mayordomo en el hogar. Pepper, desarrollado por SoftBank Robotics, est\'a dise\~nado para ser un robot de compa\~nia.\\


\subsection{RoboCup @Home League}

La RoboCup es una competencia internacional cuyo objetivo es ser un veh\'iculo para el desarrollo de la rob\'otica y la inteligencia artificial. Est\'a compuesta de variadas ligas: Rescue, Soccer, Simulation, @Home, Industrial y Junior, cada una con diversas subligas orientadas a fomentar la investigaci\'on de distintos aspectos del campo. Su sue\~no es que para mediados del siglo 21, un equipo de futbol rob\'otico completamente aut\'onomo sea capaz de vencer al campi\'on de la última copa mundial y siguiendo las reglas de la FIFA\cite{robocup:rulebook_2017}.\\ 

UChile Homebreakers participa desde el a\~no 2007 en la categor\'ia @Home. Las pruebas de la liga se desarrollan en escenarios que imitan ambientes reales, como un hogar o un restaurante. Las capacidades evaluadas y potenciadas son de Vision Computacional, Navegaci\'on aut\'onoma, Manipulaci\'on de objetos y Reconocimiento de Voz. Cada a\~no el equipo planifica sus desarrollos de acuerdo a los requerimientos de la competencia, por lo que trabajos fuera de las \'areas mencionadas no son considerados una prioridad.\\


\subsection{La Memoria Humana}

La memoria hace relaci\'on al almacenamiento de experiencias en el cerebro. Hay m\'ultiples sistemas de memoria independientes y sustentados por distintas estructuras cerebrales. A grandes rasgos, la memoria se puede dividir en de corto plazo STM (Short-Term Memory) y de largo plazo LTM (Large-Term Memory). La STM maneja informaci\'on muy detallada, es de poca capacidad y permite un r\'apido acceso, mientras que la LTM maneja mucha informaci\'on sobre experiencias y entidades, es menos detallada y de acceso m\'as lento.\\

La LTM se puede dividir en expl\'icita (consciente) e impl\'icita (inconsciente). La primera almacena datos epis\'odicos, pudiendo responder las preguntas ``Qu\'e'', ``D\'onde'' y ``Cu\'ando'', datos sem\'anticos, que modelan hechos y conceptos como el lenguaje o personas, y tambi\'en, las conexiones entre ambas submemorias. La memoria impl\'icita codifica habilidades, h\'abitos y preferencias.\\

Existen procesos de consolidaci\'on y deterioro de la memoria que est\'an constantemente en funcionamiento. La consolidaci\'on requiere un est\'imulo relevante, sumado al proceso de almacenamiento, lo que genera conexiones entre la memoria epis\'odica y la respectiva zona sem\'antica. En caso de haber experiencias repetidas, las conexiones se fortalecen. El deterioro de la memoria es un proceso que degenera las conexiones entre ambas formas de memorias expl\'icitas.\\

La memoria emocional es una forma de memoria impl\'icita que genera reacciones emocionales y sentimientos. Seg\'un los est\'imulos a los que se enfrente, permite modular el proceso de consolidaci\'on de la STM en LTM, modificando el nivel de relevancia de los eventos, pudiendo generar memorias muy fuertes y h\'abitos arraigados. Ejemplos de esto son los flashbacks y las memorias asociadas a eventos importantes\cite{Eichenbaum:2008}.\\


\section{Motivaci\'on}

La memoria es una habilidad cognitiva crucial para los humanos. Al interactuar con otras personas o el ambiente les permite recordar experiencias pasadas y sus detalles. Luego, es de esperar que un robot de servicio posea una memoria que le permita potenciar sus capacidades de interaci\'on con los humanos que ayudar\'a\cite{Vijayakumar2014}. Una LTM permitir\'ia, por ejemplo, generar di\'alogos interesantes sobre eventos pasados o cosas que el robot puede inferir del comportamiento humano, por otro lado, tambi\'en permitir\'ia la generalizaci\'on de las tareas que tiene que llevar a cabo.\\

Particularmente, dado el enfoque de las plataformas a utilizar, Bender c\'omo robot mayordomo y Pepper c\'omo robot social, se espera que ambos posean capacidades avanzadas de interaci\'on con los humanos, para lo que se requiere una LTM.\\


\subsection{Problema}

El a\~no 2015 se desarroll\'o una LTM epis\'odica para el robot Bender, orientada a la interacci\'on con personas y objetos\cite{Sanchez:2015}. El trabajo consideraba m\'etodos para almacenar, adquirir y manejar la informaci\'on epos\'odica, sumado a un proceso simple de consolidaci\'on de memoria.\\

Actualmente la memoria desarrollada no est\'a operativa, ni es factible habilitarla. A continuaci\'on se listan los aspectos que se consideran causas del problema desde un punto de vista t\'ecnico y humano:
\begin{itemize}
\item No se integr\'o adecuadamente al software del robot, no se recopila ni provee informaci\'on continuamente mientras el robot est\'a en funcionamiento.
\item La memoria no provee una API que siga el est\'andar de los desarrollos del equipo. 
\item RoboCup@Home no considera el uso de LTM en sus competencias, por lo que el equipo no tiene un incentivo real para seguir desarrollando o mantener la memoria. Esto adem\'as ha provocado que el c\'odigo quede obsoleto.
\end{itemize}

Por otro lado, suponiendo que la memoria estuviese integrada adecuadamente, existen los siguientes problemas:
\begin{itemize}
\item S\'olo considera 2 modelos sem\'anticos: Persona y Objeto, para los cuales s\'olo se almacena informaci\'on de nombre, nacionalidad e imagen.
\item A pesar de considerar un modelo para objetos, no se integr\'o con los m\'odulos relacionados que recopilan la informaci\'on, por lo que realmente la memoria s\'olo funciona para entidades de tipo Persona.
\item Es esperable que una memoria considere m\'as modelos (Personas, Objetos, Autos, Ni\~nos, Mascotas, $\ldots$) y m\'as caracter\'isticas para cada modelo (nombre, hobbies, trabajo, $\ldots$).
\item La consolidaci\'on de memoria STM a LTM s\'olo considera la primera interacci\'on con cada entidad, por lo que no existe actualizaci\'on de los datos.
\item Existe una restricci\'on en los modelos y caracter\'isticas a almacenar, respecto a la informaci\'on que el robot es realmente capaz de obtener.
\end{itemize}


\subsection{Oportunidad}

Existe un vasto desarrollo respecto a la memoria y los procesos cognitivos, sin embargo, la investigaci\'on se concentra en campos como psicolog\'ia, neurolog\'ia y ciencias cognitivas. 
Los estudios de LTM para robots de servicio son muy acotados y no existe una soluci\'on est\'andar a implementar. Algunos robots comerciales, como Pepper, utilizan LTM, pero el c\'odigo asociado no es libre, ni est\'a basado en ROS.\\

El uso de LTM no est\'a en las prioridades ``RoboCup'' del equipo, sino que es algo \'util para demostraciones y para potenciar la interacci\'on humano-robot. Por ello, se considera que no basta con desarrollar un m\'odulo capaz de recopilar informaci\'on inteligentemente, sino que adem\'as se requiere una integraci\'on con las capacidades de di\'alogo e inferencia de informaci\'on, por ejemplo, mediante la implementaci\'on de una demostraci\'on de esta habilidad.\\

As\'i, \'esta es una oportunidad para dise\~nar una memoria de largo plazo para robots de servicio, que considere aspectos como: 
\begin{itemize}
\item Implementaci\'on est\'andar y de c\'odigo libre basada en ROS
\item Memoria epis\'odica y sem\'antica adecuada a tareas generales de robots de servicio.
\item Memoria emocional que permita dar relevancia a los eventos.
\item M\'etodos de consolidaci\'on de STM en LTM
\item Recopilaci\'on continua de informaci\'on
\item Integraci\'on con el di\'alogo que realiza el robot.
\end{itemize}


\section{Posibles Soluciones}

Respecto al dise\~no de la memoria, un camino posible es modelar los datos y sus interacciones de acuerdo al funcionamiento de la memoria humana. Continuando con los trabajos de Vijayakumar\cite{Vijayakumar2014} o de Sanchez\cite{Sanchez:2015}, que desarrollan memorias epis\'odicas de largo plazo.\\

Respecto a la integraci\'on en la operaci\'on normal del robot, se puede implementar una rutina de conversaci\'on que utilice informaci\'on de la memoria. Para esto, una opci\'on es utilizar un chatbot basado en el lenguaje AIML.\\

Otro camino es realizar ingenier\'ia reversa sobre las soluciones existentes y que no son de c\'odigo libre, para entender como manejan y utilizan la memoria.\\


\section{Objetivos}

\subsection{Objetivo General}

El objetivo general corresponde al dise\~no de una memoria de largo plazo para robots de servicio, que considere componentes epis\'odicos, sem\'anticos y emocionales. La memoria debe ser integrada en Bender y Pepper, con un m\'odulo de funcionamiento continuo para recuperaci\'on y guardado de la informaci\'on, y siguiendo los est\'andares de desarrollo del equipo UChile Homebreakers. Adem\'as, se debe considerar la integraci\'on con las capacidades de di\'alogo de los robots.\\

De otra forma, el producto final es una LTM integrada en los robots, de generaci\'on continua de recuerdos y que provea una demostraci\'on de \'esta capacidad.\\


\subsection{Objetivos Espec\'ificos}

Para evitar extender innecesariamente el documento de propuesta, los objetivos espec\'ificos se presentan a modo de requisitos de usuario, los que se derivan de la motivaci\'on del trabajo. Los objetivos espec\'ificos son un desglose del objetivo general en tareas m\'as acotadas.\\

\subsubsection{De Dise\~no}

\begin{itemize}
\item Dise\~no de arquitectura de software basada en ROS y que permita que sus m\'odulos definan o dependan de modelos sem\'anticos. 
\item Definici\'on de modelos sem\'anticos de inter\'es para robots de servicio y en particular, para Bender y Pepper: Personas, Objetos, Mascotas, Mapas, ...
\item Definici\'on de caracter\'isticas de relevancia para los modelos seleccionados.
\item Definici\'on de m\'etodos de almacenamiento, formato y compresi\'on de la informaci\'on.
\item Selecci\'on/Dise\~no de algoritmo para consolidaci\'on de STM a LTM. Considerando:
\subitem - Selecci\'on de informaci\'on a almacenar y evitar redundancia.
\subitem - Cu\'ando y c\'omo actualizar la informaci\'on.
\subitem - Mecanismos de consolidaci\'on versus deterioro controlado de la informaci\'on.
\subitem - Memoria emocional, est\'imulos y modulaci\'on del proceso de consolidaci\'on.
\item Definir metodolog\'ia para generaci\'on de respaldos y recuperaci\'on de la memoria. Mezcla del respaldo con memoria ya existente.
\end{itemize}

\subsubsection{De Implementaci\'on: Core}

\begin{itemize}
\item Implementaci\'on est\'andar ROS y liberaci\'on del c\'odigo.
\item Proveer una API acorde a Bender y Pepper.
\item M\'odulo que funcione en background y que recopile informaci\'on siempre que el robot est\'e activo.
% \item Interacci\'on con la memoria a corto plazo de Bender y Pepper, como fuente principal para recopilar informaci\'on. % hay muchas fuentes de STM
\item Implementaci\'on de demostraci\'on que utilice la memoria desarrollada.
\end{itemize}

\subsubsection{De Implementaci\'on: Alternativas de desarrollo}

De manera alternativa a los aspectos considerados anteriormente, se proponen las siguientes derivaciones del trabajo, de las cuales se escoger\'a al menos 1 a modo de complemento de la soluci\'on.
\begin{itemize}
\item Inferencia de informaci\'on a partir de datos de la memoria. Por ejemplo: ``Juan suele desayunar a las 9am'', ``El control de la TV suele estar en el sof\'a'', $\ldots$.
\item Visualizador de la memoria e interfaz para facilitar gesti\'on de los recuerdos por parte de un operador.
\item Migraci\'on de memoria a otros robots.
\item Memoria compartida entre robots.
\item Uso de memoria emocional para dar personalidad y reflejar estado de \'animo.
\item Complementar recopilaci\'on de informaci\'on e inferencia mediante informaci\'on WEB.
\end{itemize}


\subsubsection{De Validaci\'on}

Orientados a guiar el desarrollo del trabajo:
\begin{itemize}
\item Definir consultas que requieran informaci\'on epis\'odica y sem\'antica.
\subitem - Epis\'odicas: ``¿Qu\'e hiciste ayer y c\'omo?'', ``Qu\'e pas\'o hace 1 mes''.
\subitem - Sem\'antica: ``¿Qu\'e ha cambiado desde tu \'ultima visita?'', ``Describe a Juan''.
\item Definir consultas que requieran informaci\'on emocional. Ejemplo: ``Enumera los 10 eventos m\'as importantes que conoces''.
\end{itemize}

Para validar la demostraci\'on de la memoria, usando las consultas definidas, se proponen las siguientes alternativas, en orden decreciente de prioridad (y creciente de complejidad):
\begin{enumerate}
\item Usar la API para responder a las preguntas definidas.
\item Mostrar la informaci\'on solicitada en el visualizador (en caso de elegir esa propuesta)
\item Uso de servicio web para an\'alisis de frases e intenciones, para responder preguntas en lenguaje natural.
\item Uso de chatbot para demostraci\'on de conversaci\'on.
\end{enumerate}


% mostrar la importancia de este tema en HRI e incentivar la consideraci\'on de esto en la competencia...\improvement{hay otras competencias que enfaticen esto??}


\section{Metodolog\'ia}

Tomando en cuenta la orientaci\'on del curso de XX6908 y XX6909, se propone una metodolog\'ia de 2 etapas.\\

Durante el primer semestre se espera:
\begin{itemize}
\item Realizar\'a una revisi\'on bibliogr\'afica, asociada a la implementaci\'on de LTM en robots de servicio y otras plataformas.
\item Dise\~no preliminar de la arquitectura de software.
\item Completar los objetivos de dise\~no que requieran la definici\'on de conceptos
\item Definici\'on de software y algoritmos a ocupar. Base de datos, consolidaci\'on, $\ldots$
\end{itemize}

Durante el segundo semestre, se propone seguir una estrategia incremental de desarrollo de software, que considere el core como primera iteraci\'on y los objetivos alternativos en las siguientes. An\'alogamente, el desarrollo del core ser\'a dividido en los siguientes incrementos m\'as peque\~nos:
\begin{enumerate}
\item Implementaci\'on de la LTM y su API.
\item M\'odulo de recopilaci\'on continua de informaci\'on
\item Implementaci\'on de la demostraci\'on.
\end{enumerate}
 
\bibliographystyle{IEEEtran}
\bibliography{IEEEabrv,bibliography.bib}

\end{document}