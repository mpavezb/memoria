
\chapter{Contextualizaci\'on}

\section{Robots de servicio}

La Federaci\'on Internacional de Rob\'otica (IFR)\cite{IFR} define \textit{robot} como:
\begin{quotation}
``Un mecanismo actuado y programable en dos o m\'as ejes y con un cierto grado de autonom\'ia, que se mueve en su entorno para realizar tareas previstas. En este contexto, autonom\'ia se refiere a la habilidad de realizar tareas previstas, basado en el estado actual y lo sensado, sin intervenci\'on humana.''
\end{quotation}

Asimismo, la IFR define un \textit{robot de servicio} como un robot ``que realiza tareas \'utiles para humanos o equipamiento, excluyendo aplicaciones de automatizaci\'on industrial''. As\'i, un robot de servicio debe trabajar en ambientes no controlados y con la autonom\'ia suficiente que le permita llevar a cabo su cometido. Generalmente, la rob\'otica de servicio se enfoca en asistir a los seres humanos en tareas repetitivas y comunes.

Seg\'un su \'area de aplicaci\'on, un rob\'ot de servicio se clasifica en \textit{de uso personal} o \textit{de uso profesional}. Los primeros son utilizados en ambientes no comerciales y por personas comunes; como por ejemplo, un robot sirviente o una silla de ruedas aut\'onoma. Un robot de servicio profesional se utiliza en ambientes comerciales, usualmente operados por alguien entrenado; un ejemplo son los robots de entrega de paquetes o para cirug\'ia.


\subsection{Robots Dom\'esticos}

Seg\'un la recopilaci\'on de datos realizada por la IFR durante el 2016, este tipo de robots es utilizado en las siguientes \'areas:
\begin{itemize}[topsep=0pt]
\setlength\itemsep{0.2em}
\item Tareas dom\'esticas: De compa\~nia, asistencia, limpieza, cuidado del hogar.
\item Entretenimiento: Juguetes, comunicaci\'on, educaci\'on e investigaci\'on.
\item Asistencia a ancianos y discapacitados: Sillas rob\'oticas y robots para cuidar personas.
\item Transporte.
\item Seguridad y vigilancia.
\item Otros que no caen en las categor\'ias anteriores.
\end{itemize}
\bigskip

El foco de este trabajo son los robots de servicio personales, dedicados a tareas dom\'esticas, clasificaci\'on a la que en  adelante se referir\'a como \textit{Robots Dom\'esticos}.

Para entender el alcance del trabajo, en cuanto a qu\'e es lo que se espera del sistema, a continuaci\'on se listan algunas capacidades de los robots dom\'esticos. Un robot de compa\~nia y asistencia tiene, pero no se limita a las siguientes tareas:
\begin{itemize}[topsep=0pt]
\setlength\itemsep{0.2em}
\item Interacci\'on amistosa con humanos.
\item Ayudar a recordar y organizar tareas.
\item Cooperar con la realizaci\'on de un procedimiento.
\item Guiar y seguir a personas.
\item Recordar informaci\'on y entidades.
\end{itemize}
\bigskip

Algunas tareas que robots dom\'esticos de tipo mayordomo deben ejecutar son:
\begin{itemize}[topsep=0pt]
\setlength\itemsep{0.2em}
\item Ofrecer comida y bebestibles.
\item Preparaci\'on de comida.
\item Ordenar y limpiar el hogar.
\end{itemize}
\bigskip


\section{Memoria y rob\'otica}

\unsure[inline]{bla bla generico}


\subsection{Trabajos relacionados}

requerimientos para interaccion humana y tareas que debe desarrollar un robot de servicio.
%\subsection{Interacci\'on Humano-Computador}
%
%\subsection{Interacci\'on con el entorno}

sobre importancia de la memoria en un sistema robotico!!!
... blabla util \cite{Vijayakumar2014}
... como memoria ayuda a mejorar desempeno en tareas \cite{Salgado2012}
... mas al respecto \cite{Ho2009}
... casos de uso e inferencia \cite{Vijayakumar2014}


diagramas
... memoria humana \cite{Vijayakumar2014}
... memoria humana LTM \cite{Stachowicz2012}
... consolidacion y olvido \cite{Deutsch2008}

Punto de vista de las ciencias cognitivas
... agregar!!! o mover a la memoria humana
... referencias a psicoanalisis \cite{Deutsch2008}

Sobre casos quie no son human like...
- algunos solo base de datos y recordar todo
- deep neural network... \cite{KimMinJoo2016}


Muchos intentar basarse en bla...


Relacion memoria humana con robotica.
... habilidades y PM \cite{Salgado2012} ... CRAM
... datos y semantica
... sobre importancia del sueno y postprocesamiento \cite{Kelley2014}
 ... E LTM no solo responde que donde y cuando.. necesita perspectiva \cite{Stachowicz2012}
... S Memory.. se abstrae de perspecttiva y cosas situacionales \cite{Stachowicz2012}
... defs de memoria \cite{Deutsch2008}
... definiciones de procesos cognitivos: \cite{Deutsch2008}
 
 
Existen los siguientes modelos
- algunos solo procuran desarrollar reglas sobre como actualizar los pesos de aprendizaje
- dejar de aprender (sorprenderse al ser mayor)
... modelo probabilistico para consolidar y decaer \cite{Dodd2005}
... olvidar cosas.. si no se implementa, entonces la busqueda de informacion seria cada vez mas compleja. \cite{Deutsch2008}
... modelo capaz de adaptarse a m\'as robots \cite{Ho2009}
... modelo de contexto \cite{Sanchez:2015} FELIX

sobre el diseno..
... que recordar y cuando \cite{Kasap2010}
... conflicto etico de almacenar datos de usuarios \cite{Ho2009}
... aspectos de diseno \cite{Ho2009}
... propiedades de LTM \cite{Jockel2008}
... requerimientos de LTM \cite{Vijayakumar2014}
... requerimientos E LTM \cite{Stachowicz2012}
... 11 req. ELTM \cite{Stachowicz2012}
... sobre importancia del sueno y postprocesamiento \cite{Kelley2014}
... diseno explicado de SONIA en RDF \cite{Vijayakumar2014}
... definir contenido de QWHAT \cite{Stachowicz2012}

Existen los siguientes frameworks, modelos
- procedural y CRAM \cite{Winkler2014}
... ISAC \cite{Dodd2005}
... MINERVA, LIDA, Neuronal,M SMRTI \cite{Jockel2008}
... deficiencias de ISAC, EPIROME, \cite{Stachowicz2012}
... sobre Tecuci, ISAC, SOAR y Ho \cite{Deutsch2008}

Existen las siguientes reglas generales
.. definicion de episodio \cite{Dodd2005}
.. reglas de consolidacion \cite{Dodd2005}


sobre memoria emocional y mapeo al robot
.. emotion Engine \cite{Kasap2010}
.. mapeo por dolor. teoria de haikonen y otros \cite{Dodd2005}
... refs sobre emociones \cite{Deutsch2008}
.. emociones modulan el bla \cite{Deutsch2008}

sobre inferencia y mapeo al robot
.. sobre importancia del sueno y postprocesamiento \cite{Kelley2014}

Resumen de dificultades ... diagrama o tabla.



Considerar unificaci\'on de informaci\'on.. que pasa si almaceno obj1 y obj2, pero luego aprendo que son el mismo?



\section{Componentes de software}

La implementaci\'on de un sistema de memoria ro\'otica asume que muchos sistemas y capacidades de un robot est\'an disponibles. Entonces, se requiere del uso de variados frameworks y librer\'ias que permiten comunicarse con el robot y acceder a los datos de inter\'es que se desea recordar. A continuaci\'on se explican los componentes de software relevantes para el trabajo.

\subsection{ROS}

Los sistemas rob\'oticos actuales son cada vez m\'as complejos. Deben lidiar con muchos componentes tanto de hardware como de software y su interacci\'on, de una forma eficaz y que no entorpezca el desarrollo. Muchas tareas de control requieren altas frecuencias de funcionamiento, as\'i como la sincronizaci\'on y comunicaci\'on entre los diversos m\'odulos. Por lo tanto, el c\'omo se unen los subsistemas en una aplicaci\'on rob\'otica es una tarea dif\'icil.

ROS\cite{ROS:2009}, acr\'onimo para Robot Operating System, es un proyecto que funciona como middleware para aplicaciones rob\'oticas, y permite resolver el problema de la comunicaci\'on entre procesos. Es una colecci\'on de herramientas, librer\'ias y convenciones que buscan simplificar la tarea de crear comportamientos rob\'oticos complejos y robustos, sin importar la plataforma rob\'otica.

Fue originalmente creado por la empresa WillowGarage en el 2008, y mantenido actualmente por la Open Source Robotics Foundation (OSRF). Existe un ecosistema ROS, mantenido por la comunidad, y con cientos de m\'odulos de software con soluciones a problemas espec\'ificos, los que pueden interconectarse para construir comportamientos m\'as complejos. Por lo anterior, su uso se ha convertido en una pr\'actica mundial, siendo adoptado incluso en soluciones industriales.

\todo[inline]{M\'as informaci\'on sobre ROS. Lo que sea relevante para la implementaci\'on y el dise\~no}

%\subsubsection{Nodo}
%
%Mensajes y master
%
%\subsubsection{T\'opicos}
%
%\subsubsection{Servicios}
%
%\subsubsection{Servidor de par\'ametros}
%
%\subsubsection{Launch}

%\subsubsection{Herramientas}

%\subsubsection{Actionlib}


%\subsection{SMACH}

\todo[inline]{Sobre SMACH Ser\'a utilizado para la demo.}


\subsection{UChile ROS Framework}

UChile ROS Framework (URF) hace referencia al sistema de software desarrollado en el laboratorio de rob\'otica del Departamento de Ingenier\'ia El\'ectrica de la Universidad de Chile, para sus robots de servicio. El sistema cuenta con 10 a\~nos de desarrollo y est\'a orientado a cumplir los requisitos de la competencia Robocup en su categor\'ia @Home.

URF est\'a construido sobre ROS y en una estructura de 4 capas. La primera capa contiene todas las dependencias del sistema, ya sean de ROS o no; Es la \'unica capa donde que contiene c\'odigo externo. Sobre ella, se monta una capa ROS de bajo nivel, con herramientas y librer\'ias comunes, sumado a los drivers necesarios para manejar cada robot. La capa intermedia alberga capacidades rob\'oticas avanzadas, relacionadas a percepci\'on ro\'otica, manipulaci\'on de objetos, navegaci\'on aut\'onoma e interacci\'on Humano-Robot. Finalmente, existe una capa desarrollada en python, con interfaces para el uso de las capacidades de menor nivel, utilizada para la elaboraci\'on de maquinas de estado y comportamientos rob\'oticos complejos.

\todo[inline]{Agregar imagen con las capas de URF}

Todos los m\'odulos de URF son de c\'odigo libre, a excepci\'on de los algoritmos relacionados con percepci\'on y la interfaz de alto nivel. El c\'odigo se almacena p\'ublicamente en la organizaci\'on \textit{uchile-robotics} en GitHub\footnote{Organizaci\'on \textit{uchile-robotics} y URF en GitHub: \url{https://github.com/uchile-robotics}}.


%\subsubsection{Concepto de robot-skills}
\todo[inline]{Sobre las robot skills. Ser\'an utilizadas para la demo.}


\subsubsection{Manejo de informaci\'on en URF}

Si un robot implementa URF, entonces es posible acceder a la informaci\'on compartida por sus procesos. Cualquier m\'odulo ROS en el sistema tiene acceso a los datos extra\'idos desde sensores y luego generados en postprocesamientos, junto al acceso para controlar el hardware.

Existen algunas formas de memoria implementadas en URF, comparables a los conceptos definidos para la memoria humana. Tambi\'en se pueden dividir en de corto y largo plazo:

Como STM, se puede definir como memoria de trabajo a todo el flujo de informaci\'on presente durante la ejecuci\'on del robot. Lo que incluye datos sensados, procesamientos y acciones realizadas. Generalmente tales datos no son almacenados para posteriores ejecuciones.

A manera de LTM, se puede encontrar una memoria procedural, relacionada con todo el conocimiento almacenado que posee el robot para cumplir ciertas tareas. Caen en esta categor\'ia: modelos para percepci\'on rob\'otica, modelos para reconocimiento de voz y patrones, bases de datos de movimientos precalculados para manipular objetos y acciones predefinidas que se utilizan para controlar el robot.

Tambi\'en se pueden encontrar especializaciones de memoria LTM sem\'antica. Ejemplos de \'esto son: El m\'apa que se conoce del entorno, junto a los lugares y objetos anotados en \'el. Diccionarios con informaci\'on anotada sobre entidades y sus caracter\'isticas, c\'omo personas y objetos. Bases de datos con im\'agenes anotadas para el reconocimiento de objetos y personas. 

Sin embargo, en URF no existen formas de memoria emocional ni epis\'osica de largo plazo. Luego, toda interacci\'on realizada por los robots est\'a limitada a la informaci\'on obtenida desde el inicio al t\'ermino de cada rutina.



\subsection{KnowRob}

Permite implementar memoria sem\'antica (facts) y procedural (actions).
\cite{Tenorth2013}, \cite{Tenorth2009}

%\subsection{MongoDB}
%
%

\subsection{Bases de Datos}

Base Relacional SQLite Para memoria Epis\'odica. Es realmente necesario?.. quizas basta con MongoDB

Base No Relacional Mongo: para datos

OWL para sem\'antica. Pues permite realizar inferencias.


Son estructuras "cerebrales especializadas" para las tareas que deben cubrir.




\section{Plataformas objetivo}

\subsection{Robot Bender}

Bender es un robot humanoide creado el a\~no 2007 en el laboratorio de rob\'otica del Departamento de Ingenier\'ia El\'ectrica de la Universidad de Chile. El equipo UChile Homebreakers es el encargado de su desarrollo y  su objetivo es ser un mayordomo para el hogar, funcionando de manera aut\'onoma para apoyar en tales labores\cite{uchile-robotics}.

%\todo[inline]{FOTO}

En cuanto a actuadores, el robot cuenta con 3 brazos antropom\'orficos de 6 grados de libertad cada uno, una base m\'ovil diferencial Pioneer 3-AT, un cuello que permite rotaciones en dos ejes cartesianos; pudiendo imitar gestos de asentimiento y negaci\'on, y finalmente, una cabeza que puede mostrar expresiones faciales mediante movimientos de su boca, orejas, cejas y cambios de colores alrededor de los ojos.

El robot cuenta con los siguientes sensores: un laser Hokuyo UTM-30LX, un laser Hokuyo URG-04LX-UG01, un micr\'ofono M-Audio Producer USB y una c\'amara de profundidad ASUS Xtion Pro.

El software de Bender est\'a basado en el framework URF. Su arquitectura de software utiliza  ROS para el manejo de componentes de bajo y medio nivel. La capa de alto nivel, escrita en python, se abstrae de ROS y permite la creaci\'on de comportamientos complejos mediante m\'aquinas de estado. Todos los m\'odulos que interactuan con sensores y actuadores est\'an implementados en ROS.


%\subsection{Pepper}
%
%Pepper es un robot humanoide desarrollado por SoftBank Robotics, orientado a la interacci\'on humano robot y con el objetivo de ser un robot de compa\~nia y soporte emocional.


\section{Contribuci\'on del Trabajo}



