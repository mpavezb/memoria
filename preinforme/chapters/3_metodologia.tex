
\chapter{Metodología}

En este capítulo se presenta la metodología a utilizar para el desarrollo del trabajo de título. En primer lugar se presentan las alternativas de solución evaluadas para el proyecto, haciendo énfasis en la factibilidad de la propuesta. Para luego, describir la planificación de tareas a realizar durante el segundo periodo del proceso, sustentados en las soluciones propuestas. Finalmente, se presenta una carta gantt con los periodos de trabajo estimados.


\section{Propuesta de Solución}

%Temas a resolver:
%
%- modelo de memoria a utilizar
%- arquitectura
%- consolidación
%- base de datos
%- demostracion 
%- emociones
%- inferencia


El diseño de la memoria LTM estará basado en los 11 requerimientos de diseño mostrados en la sección \ref{sec:ltm_exp}. Además se basará en la taxonomía de la memoria humana, de manera similar a los trabajos desarrollados por Vijayakumar\cite{Vijayakumar2014} y Sánchez et al.\cite{Sanchez:2015}.

En cuanto a los modelos semánticos a considerar y sus características, se ha acotado el tema a robots de servicio de tipo doméstico, pues son los disponibles en la Universidad. Particularmente, el foco del trabajo, será generar una memoria LTM adecuada al robot Bender. Falta definir exactamente las consultas que debe ser capaz de manejar el robot, según los requerimientos para robots domésticos mostrados en la sección \ref{sec:domestic_robots}

El algoritmo a utilizar para la consolidación de STM en LTM no está claro aún y debe ser estudiado. No existe un consenso al respecto en la literatura, pero un primer acercamiento será basado en la propuesta de Sánchez et al.\cite{Sanchez:2015}.

Una vez definido el proceso de consolidación, queda definir una arquitectura de software. La intención es basar el desarrollo en el software KnowRob, pues provee muchas de las funcionalidades requeridas para las memorias semántica y procedural. El desafío se centra en poder agregar el soporte para memoria episódica al sistema. KnowRob ha sido descargado y se ha comprobado el funcionamiento de sus módulos principales.

Tras implementar lo anterior, se debe implementar un servicio que escuche constantemente los eventos ocurridos en el sistema. Luego se debe implementar una demostración del uso de los datos recopilados.

En cuanto a la inferencia de información, se aprovechará que KnowRob ya provee herramientas para ello. La memoria emocional sólo será considerada como un objetivo secundario.

\todo[inline]{Sobre como se puede implementar la memoria emocional a partir de mediciones simples en el robot.}

\todo[inline]{Sobre como será la rutina de desmostración a implementar.}

% Una vez definido el proceso de consolidación, se puede implementar el servicio que recopile información constantemente. éste podría acceder a la información que genera el robot durante sus rutinas, una forma de STM básica, mediante sus interfaces ROS.

% \item Interacción con la STM de Bender y Pepper, como fuente principal para recopilar información. % hay muchas fuentes de STM

% Respaldos
%Capacidad de generar respaldos de la memoria y recuperación de éstos. Mezcla del respaldo con memoria ya existente.

%Respecto a la integración en la operación normal del robot, se puede implementar una rutina de conversación que utilice información de la memoria. Para esto, una opción es utilizar un chatbot basado en el lenguaje AIML\cite{aiml}.
%\item Implementación de demostración que utilice la memoria desarrollada.

% Para validar el diseño e implementación de la LTM, algunos procedimientos de interés son:
%\begin{itemize}
%\item Definición de consultas que requieran información episódica y semántica, para ser respondidas por la LTM. Por ejemplo:
%\subitem - Consultas episódicas: ``Qué hiciste ayer y cómo?'', ``Qué pasó hace 1 mes?''.
%\subitem - Consultas semánticas: ``Qué ha cambiado en la habitación?'', ``Describe un humano que conozcas.''.
%\item Uso de la API para responder a las preguntas definidas.
%\item Uso de chatbot básico basado en AIML\cite{aiml}, a modo de demostración de conversación.
%\item Queda definir otras alternativas de validación
%\end{itemize}

% En cuanto a la memoria emocional, hay dos partes de interés, asignar importancia a los eventos y usar ese indicador para modular el proceso de consolidación. Ambos componentes deben ser estudiados con más detalle. Para su validación bastaría definir consultas que requieran información emocional, como por ejemplo: ``Enumera los 10 eventos más importantes que conoces''.


%\subsubsection{OTROS}
%\begin{itemize}
%\item Visualizador de la memoria e interfaz para facilitar gestión de los recuerdos por parte de un operador.
%\item Migración de memoria a otros robots.
%\item Memoria compartida entre robots.
%\item Uso de memoria emocional para dar personalidad y reflejar estado de ánimo.
%\item Complementar recopilación de información e inferencia mediante información WEB.
%\item Uso de servicio web para análisis de frases e intenciones, para responder preguntas en lenguaje natural.
%\end{itemize}


\section{Planificación}

Durante el trabajo de título se propone seguir una estrategia incremental de desarrollo de software, que considere el \textit{core} como primera iteración y los 2 objetivos alternativos en las siguientes. El desarrollo del core será dividido en los siguientes incrementos más pequeños:
\begin{enumerate}[topsep=0pt]
\setlength\itemsep{0.2em}
\item Definir consultas a responder.

\item Definir algoritmo de consolidación.

\item Diseño de la arquitectura de software basada en KnowRob y ROS.

\item Implementación de la Ep-LTM en KnowRob.

\item Servicio para recopilación continua de información.

\item Implementación de la demostración.
\end{enumerate}


\subsection{Carta Gantt para el Trabajo de Título}


\begin{center}
\scalebox{1.2}[1.0]{
%\rotatebox{0}{
\boxed{
\begin{gantt}{10}{9}
	\begin{ganttitle}
		\titleelement{Ago.}{1}
		\titleelement{Sept.}{2}
		\titleelement{Oct.}{2}
		\titleelement{Nov.}{2}
		\titleelement{Dic.}{2}
    \end{ganttitle}
    \ganttbar[color=cyan]{Consultas}{0}{1}
    \ganttbar[color=cyan]{Consolidación}{0}{1}
    \ganttbar[color=red]{Diseño}{1}{1}
    \ganttbar[color=magenta]{Implementación}{2}{2}
    \ganttbar[color=yellow]{Integración}{4}{1}
	\ganttbar[color=red]{Demostración}{5}{1}
	\ganttbar[color=red]{Inferencia}{6}{1}
	\ganttbar[color=red]{M. Emocional}{7}{1}
	\ganttbar[color=gray]{Escritura}{0}{9}
\end{gantt}
%}
}}
\end{center}

