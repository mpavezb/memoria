
\chapter{Metodología}\label{chapter:metodologia}

En este capítulo se presenta la metodología a utilizar para el desarrollo del trabajo de título. En primer lugar se presentan las alternativas de solución evaluadas para el proyecto, haciendo énfasis en la factibilidad de la propuesta. Para luego, describir la planificación de tareas a realizar durante el segundo periodo del proceso, sustentadas en las soluciones propuestas. Finalmente, se presenta una carta gantt con los periodos de trabajo estimados.


%\section{Requerimientos para un robot doméstico}
%
%\todo[inline]{Sobre los requerimientos en específico para robots domésticos}
%
%Para entender el alcance del trabajo, en cuanto a qué es lo que se espera del sistema, a continuación se listan algunas capacidades de los robots domésticos. Un robot de compañía y asistencia tiene, pero no se limita a las siguientes tareas:
%\begin{itemize}[topsep=0pt]
%\setlength\itemsep{0.2em}
%\item Interacción amistosa con humanos.
%\item Ayudar a recordar y organizar tareas.
%\item Cooperar con la realización de un procedimiento.
%\item Guiar y seguir a personas.
%\item Recordar información y entidades.
%\end{itemize}
%\bigskip
%
%Algunas tareas que robots domésticos de tipo mayordomo deben ejecutar son:
%\begin{itemize}[topsep=0pt]
%\setlength\itemsep{0.2em}
%\item Ofrecer comida y bebestibles.
%\item Preparación de comida.
%\item Ordenar y limpiar el hogar.
%\end{itemize}
%\bigskip

%\subsubsection{Requerimientos para robots domésticos}
%
%En términos de LTM, un robot doméstico necesita poder recordar los siguientes conocimientos:
%
%... \todo[inline]{tareas de un robot de servicio}
%\begin{itemize}
%\item 
%\end{itemize}
%
%... casos de uso e inferencia \cite{Vijayakumar2014}
%
%... artificial companion PAG 13 \cite{Vijayakumar2014}
%


\section{Propuesta de Solución}

%Temas a resolver:
%
%- modelo de memoria a utilizar
%- arquitectura
%- consolidación
%- base de datos
%- demostracion 
%- emociones
%- inferencia


El diseño de la memoria LTM estará basado en los 11 requerimientos de diseño mostrados en la sección \ref{sec:ltm_exp}. Además se basará en la taxonomía de la memoria humana, de manera similar a los trabajos desarrollados por Vijayakumar  \cite{Vijayakumar2014} y Sánchez et al. \cite{Sanchez:2015}.

En cuanto a los modelos semánticos a considerar y sus características, se ha acotado el tema a robots de servicio de tipo doméstico, pues son los disponibles en la Universidad. Particularmente, el foco del trabajo, será generar una memoria LTM adecuada al robot Bender. Falta definir exactamente las consultas que debe ser capaz de manejar el robot, según los requerimientos para robots domésticos mostrados en la sección \ref{sec:domestic_robots}.

El algoritmo a utilizar para la consolidación de STM en LTM no está claro aún y debe ser estudiado. Esta es la parte más relevante del trabajo, donde se tomarán las decisiones de diseño definitivas. No existe un consenso al respecto en la literatura, pero un primer acercamiento será basado en la propuesta de Sánchez et al.  \cite{Sanchez:2015}, para la definición de episodios, sumado al trabajo de Kelley  \cite{Kelley2014} para la selección de qué eventos almacenar.

Una vez definido el proceso de consolidación, queda definir una arquitectura de software. La intención es basar el desarrollo en el software KnowRob, pues fue diseñado para manejar memoria semántica y procedural, además de proveer funcionalidades para realizar inferencias. Por su estructura, se cree que la Ep-LTM puede ser almacenada como si fuese memoria semántica. Entonces, el desafío se centra en poder agregar el soporte para memoria episódica al sistema. KnowRob ha sido descargado y se ha comprobado el funcionamiento de sus módulos principales.

Tras implementar lo anterior, se debe implementar un servicio que escuche constantemente los eventos ocurridos en el sistema. Luego se debe implementar una demostración del uso de los datos recopilados.

No se trabajará en la inferencia de información, sino que sólo se aprovechará de que KnowRob provee herramientas para ello, lo que asegura que en un futuro se podrán agregar inferencias al sistema. La memoria emocional sólo será considerada para la etapa de diseño, pero su implementación no es requerida.


La demostración será definida como un conjunto de validaciones que permitan corroborar el funcionamiento del sistema. Por lo tanto, esta será implementada incrementalmente, para validar cada desarrollo del trabajo. Las validaciones más relevantes tienen que ver con la capacidad del sistema para responder las consultas episódicas seleccionadas para robots domésticos.

%\subitem - Consultas episódicas: ``Qué hiciste ayer y cómo?'', ``Qué pasó hace 1 mes?''.
%\subitem - Consultas semánticas: ``Qué ha cambiado en la habitación?'', ``Describe un humano que conozcas.''.


% Una vez definido el proceso de consolidación, se puede implementar el servicio que recopile información constantemente. éste podría acceder a la información que genera el robot durante sus rutinas, una forma de STM básica, mediante sus interfaces ROS.

% Respaldos
%Capacidad de generar respaldos de la memoria y recuperación de éstos. Mezcla del respaldo con memoria ya existente.


%\subsubsection{OTROS}
%\begin{itemize}
%\item Visualizador de la memoria e interfaz para facilitar gestión de los recuerdos por parte de un operador.
%\item Migración de memoria a otros robots.
%\item Memoria compartida entre robots.
%\item Uso de memoria emocional para dar personalidad y reflejar estado de ánimo.
%\item Complementar recopilación de información e inferencia mediante información WEB.
%\item Uso de servicio web para análisis de frases e intenciones, para responder preguntas en lenguaje natural.
%\end{itemize}


\section{Planificación}

Durante el trabajo de título se propone seguir una estrategia incremental de desarrollo de software, que considere cada uno de los objetivos específicos. La demostración es de carácter transversal al proyecto, pues se utilizará como medio de validación de cada parte del trabajo:
\begin{enumerate}[topsep=0pt]
\setlength\itemsep{0.2em}
\item Definir consultas a responder.

\item Diseño de algoritmo de consolidación.

\item Diseño de la arquitectura de software basada en KnowRob y ROS.

\item Implementación de la Ep-LTM en KnowRob.

\item Servicio para recopilación continua de información.

\end{enumerate}

\begin{center}
\scalebox{1.2}[1.0]{
%\rotatebox{0}{
\boxed{
\begin{gantt}{8}{9}
	\begin{ganttitle}
		\titleelement{Ago.}{1}
		\titleelement{Sept.}{2}
		\titleelement{Oct.}{2}
		\titleelement{Nov.}{2}
		\titleelement{Dic.}{2}
    \end{ganttitle}
    \ganttbar[color=cyan]{Consultas}{0}{1}
    \ganttbar[color=cyan]{Consolidación}{1}{3}
    \ganttbar[color=red]{Arquitectura}{2}{2}
    \ganttbar[color=magenta]{Implementación}{4}{2}
    \ganttbar[color=yellow]{Integración}{6}{3}
	\ganttbar[color=orange]{Demostración}{0}{9}
	\ganttbar[color=gray]{Escritura}{0}{9}
\end{gantt}
%}
}}
\end{center}

