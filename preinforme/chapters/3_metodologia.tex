
\chapter{Metodolog\'ia}

En este cap\'itulo se presenta la metodolog\'ia a utilizar para el desarrollo del trabajo de t\'itulo. En primer lugar se presentan las alternativas de soluci\'on evaluadas para el proyecto, haciendo \'enfasis en la factibilidad de la propuesta. Para luego, describir la planificaci\'on de tareas a realizar durante el segundo periodo del proceso, sustentados en las soluciones propuestas. Finalmente, se presenta una carta gantt con los periodos de trabajo estimados.


\section{Propuesta de Soluci\'on}

%Temas a resolver:
%
%- modelo de memoria a utilizar
%- arquitectura
%- consolidaci\'on
%- base de datos
%- demostracion 
%- emociones
%- inferencia


El dise\~no de la memoria LTM estar\'a basado en los 11 requerimientos de dise\~no mostrados en la secci\'on \ref{sec:ltm_exp}. Adem\'as se basar\'a en la taxonom\'ia de la memoria humana, de manera similar a los trabajos desarrollados por Vijayakumar\cite{Vijayakumar2014} y Sanchez et al.\cite{Sanchez:2015}.

En cuanto a los modelos sem\'anticos a considerar y sus caracter\'isticas, se ha acotado el tema a robots de servicio de tipo dom\'estico, pues son los disponibles en la Universidad. Particularmente, el foco del trabajo, ser\'a generar una memoria LTM adecuada al robot Bender. Falta definir exactamente las consultas que debe ser capaz de manejar el robot, seg\'un los requerimientos para robots dom\'esticos mostrados en la secci\'on \ref{sec:domestic_robots}

El algoritmo a utilizar para la consolidaci\'on de STM en LTM no est\'a claro a\'un y debe ser estudiado. No existe un consenso al respecto en la literatura, pero un primer acercamiento ser\'a basado en la propuesta de Sanchez et al.\cite{Sanchez:2015}.

Una vez definido el proceso de consolidaci\'on, queda definir una arquitectura de software. La intenci\'on es basar el desarrollo en el software KnowRob, pues provee muchas de las funcionalidades requeridas para las memorias sem\'antica y procedural. El desaf\'io se centra en poder agregar el soporte para memoria epis\'odica al sistema. KnowRob ha sido descargado y se ha comprobado el funcionamiento de sus m\'odulos principales.

Tras implementar lo anterior, se debe implementar un servicio que escuche constantemente los eventos ocurridos en el sistema. Luego se debe implementar una demostraci\'on del uso de los datos recopilados.

En cuanto a la inferencia de informaci\'on, se aprovechar\'a que KnowRob ya provee herramientas para ello. La memoria emocional s\'olo ser\'a considerada como un objetivo secundario.

% Una vez definido el proceso de consolidaci\'on, se puede implementar el servicio que recopile informaci\'on constantemente. \'Este podr\'ia acceder a la informaci\'on que genera el robot durante sus rutinas, una forma de STM b\'asica, mediante sus interfaces ROS.

% \item Interacci\'on con la STM de Bender y Pepper, como fuente principal para recopilar informaci\'on. % hay muchas fuentes de STM

% Respaldos
%Capacidad de generar respaldos de la memoria y recuperaci\'on de \'estos. Mezcla del respaldo con memoria ya existente.

%Respecto a la integraci\'on en la operaci\'on normal del robot, se puede implementar una rutina de conversaci\'on que utilice informaci\'on de la memoria. Para esto, una opci\'on es utilizar un chatbot basado en el lenguaje AIML\cite{aiml}.
%\item Implementaci\'on de demostraci\'on que utilice la memoria desarrollada.

% Para validar el dise\~no e implementaci\'on de la LTM, algunos procedimientos de inter\'es son:
%\begin{itemize}
%\item Definici\'on de consultas que requieran informaci\'on epis\'odica y sem\'antica, para ser respondidas por la LTM. Por ejemplo:
%\subitem - Consultas epis\'odicas: ``Qu\'e hiciste ayer y c\'omo?'', ``Qu\'e pas\'o hace 1 mes?''.
%\subitem - Consultas sem\'anticas: ``Qu\'e ha cambiado en la habitaci\'on?'', ``Describe un humano que conozcas.''.
%\item Uso de la API para responder a las preguntas definidas.
%\item Uso de chatbot b\'asico basado en AIML\cite{aiml}, a modo de demostraci\'on de conversaci\'on.
%\item Queda definir otras alternativas de validaci\'on
%\end{itemize}

% En cuanto a la memoria emocional, hay dos partes de inter\'es, asignar importancia a los eventos y usar ese indicador para modular el proceso de consolidaci\'on. Ambos componentes deben ser estudiados con m\'as detalle. Para su validaci\'on bastar\'ia definir consultas que requieran informaci\'on emocional, como por ejemplo: ``Enumera los 10 eventos m\'as importantes que conoces''.


%\subsubsection{OTROS}
%\begin{itemize}
%\item Visualizador de la memoria e interfaz para facilitar gesti\'on de los recuerdos por parte de un operador.
%\item Migraci\'on de memoria a otros robots.
%\item Memoria compartida entre robots.
%\item Uso de memoria emocional para dar personalidad y reflejar estado de \'animo.
%\item Complementar recopilaci\'on de informaci\'on e inferencia mediante informaci\'on WEB.
%\item Uso de servicio web para an\'alisis de frases e intenciones, para responder preguntas en lenguaje natural.
%\end{itemize}


\section{Planificaci\'on}

Durante el trabajo de t\'itulo se propone seguir una estrategia incremental de desarrollo de software, que considere el \textit{core} como primera iteraci\'on y los 2 objetivos alternativos en las siguientes. El desarrollo del core ser\'a dividido en los siguientes incrementos m\'as peque\~nos:
\begin{enumerate}[topsep=0pt]
\setlength\itemsep{0.2em}
\item Definir consultas a responder.

\item Definir algoritmo de consolidaci\'on.

\item Dise\~no de la arquitectura de software basada en KnowRob y ROS.

\item Implementaci\'on de la Ep-LTM en KnowRob.

\item Servicio para recopilaci\'on continua de informaci\'on.

\item Implementaci\'on de la demostraci\'on.
\end{enumerate}


\subsection{Carta Gantt para el Trabajo de T\'itulo}


\begin{center}
\scalebox{1.5}[1.1]{
%\rotatebox{0}{
\boxed{
\begin{gantt}{8}{9}
	\begin{ganttitle}
		\titleelement{Ago.}{1}
		\titleelement{Sept.}{2}
		\titleelement{Oct.}{2}
		\titleelement{Nov.}{2}
		\titleelement{Dic.}{2}
    \end{ganttitle}
    \ganttbar[color=cyan]{Consultas}{0}{1}
    \ganttbar[color=cyan]{Consolidaci\'on}{0}{1}
    \ganttbar[color=red]{Dise\~no}{1}{2}
    \ganttbar[color=magenta]{Implementaci\'on}{3}{2}
    \ganttbar[color=yellow]{Integraci\'on}{5}{2}
	\ganttbar[color=red]{Demostraci\'on}{7}{2}
	\ganttbar[color=gray]{Escritura}{0}{9}
\end{gantt}
%}
}}
\end{center}

