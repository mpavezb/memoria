
\chapter{Introducción}

\todo[inline]{Escribir introducción de la intro.}

\section{Antecedentes Generales}

A continuación se presenta una breve introducción a los temas requeridos para contextualizar este trabajo de título: La robótica de servicio y el equipo de trabajo donde se implantará la solución. Además, se introduce el tema de la memoria humana, requerido para entender la propuesta y su relación con la robótica.


\subsection{Robots de servicio}

La robótica de servicio es un área enfocada en asistir a los seres humanos en tareas repetitivas y comunes, como la recolección de basura. Formalmente, se define un \textit{robot de servicio} como un robot ``que realiza tareas útiles para humanos o equipamiento, excluyendo aplicaciones de automatización industrial''\cite{IFR}. Luego, el robot requiere cierto grado de autonomía, que es la habilidad de actuar a partir del estado actual, usando lo que observa del ambiente y sin intervención humana. Así, un robot de servicio debe trabajar en ambientes no controlados y con la autonomía suficiente que le permita llevar a cabo su cometido.

Estos pueden se categorizar en robots para transporte, seguridad y domésticos, entre otros. Los robots de servicio domésticos se caracterizan por tareas de asistencia en el hogar y de compañía para humanos. Tareas típicas que deben realizar son ayudar a ordenar, preparar comida u ofrecer bebestibles. Otros tipos de robots se enfocan en el cuidado de adultos mayores, mascotas robots, salud o educación. Particularmente, la compañía SoftBank Robotics es pionera en ofrecer a Pepper como el primer robot humanoide ya adoptado en hogares de Japón, así como robot de bienvenida en hoteles y tiendas\cite{softbank}.


\subsection{Equipo de Trabajo: UChile Homebreakers}

El laboratorio de robótica del Departamento de Ingeniería Eléctrica de la Universidad de Chile alberga dos equipos de robótica: \textit{UChile Robotics Team}, dedicado al fútbol robótico y \textit{UChile Homebreakers Team}, enfocado en robótica de servicio. Ambos son conformados por alumnos de pregrado y postgrado de diversas especialidades, y liderados por el profesor Javier Ruiz del Solar\cite{uchile-robotics}.

UChile Homebreakers existe desde el año 2007 y actualmente cuenta con 15 estudiantes. Todo su desarrollo de software está basado en ROS, un framework para el desarrollo de plataformas robóticas y con miles de usuarios alrededor del mundo\cite{ROS:2009}.

El equipo trabaja en dos plataformas humanoides de tipo doméstico, Bender y Pepper. Bender es un robot construido en el mismo laboratorio y con el objetivo de ser un mayordomo para el hogar. Pepper, desarrollado por SoftBank Robotics, está diseñado para ser un robot de compañía. Ambos comparten la misma arquitectura de software y prácticamente todo su código, exceptuando los drivers para acceder al hardware respectivo.


\subsubsection{RoboCup @Home League}


La RoboCup es una competencia internacional cuyo objetivo es ser un vehículo para el desarrollo de la robótica y la inteligencia artificial. Está compuesta de variadas ligas: Rescue, Soccer, Simulation, @Home, Industrial y Junior, cada una con diversas subligas orientadas a fomentar la investigación de distintos aspectos del campo. Su sueño es que para mediados del siglo 21, un equipo de fútbol robótico completamente autónomo sea capaz de vencer al campión de la última copa mundial y siguiendo las reglas de la FIFA\cite{robocup:rulebook_2017}.

UChile Homebreakers participa desde el año 2007 en la categoría @Home. Las pruebas de la liga se desarrollan en escenarios que imitan ambientes reales, como un hogar o un restaurante. 
%Además, la competencia funciona como un espectáculo para público general, por lo que se priorizan pruebas y demostraciones interesantes para los espectadores.
% 
Las capacidades generalmente evaluadas y potenciadas en @Home son de Visión Computacional, Navegación autónoma, Manipulación de objetos y Reconocimiento de Voz. Cada año el equipo planifica sus desarrollos de acuerdo a los requerimientos de la competencia, por lo que trabajos fuera de las áreas mencionadas no son considerados una prioridad.


\subsection{La memoria humana}

La memoria hace relación al almacenamiento de experiencias en el cerebro. Hay múltiples sistemas de memoria independientes y sustentados por distintas estructuras cerebrales. A grandes rasgos, la memoria se puede dividir en de corto plazo STM (Short-Term Memory) y de largo plazo LTM (Large-Term Memory). La STM maneja información muy detallada, es de poca capacidad y permite un rápido acceso, mientras que la LTM maneja mucha información sobre experiencias y entidades, es menos detallada y de acceso más lento\cite{Eichenbaum:2008}.

La LTM se puede dividir en explícita (consciente) e implícita (inconsciente). La primera almacena datos episódicos, pudiendo responder las preguntas ``Qué'', ``Dónde'' y ``Cuándo'', datos semánticos, que modelan hechos y conceptos como el lenguaje o personas, y también, las conexiones entre ambas submemorias. La memoria implícita codifica habilidades, hábitos y preferencias.

Existen procesos de consolidación y deterioro de la memoria que están constantemente en funcionamiento. La consolidación requiere un estímulo relevante, sumado al proceso de almacenamiento, lo que genera conexiones entre la memoria episódica y la respectiva zona semántica. En caso de haber experiencias repetidas, las conexiones se fortalecen. El deterioro de la memoria es un proceso que degenera las conexiones entre ambas formas de memorias explícitas.

La memoria emocional es una forma de memoria implícita que genera reacciones emocionales y sentimientos. Según los estímulos a los que se enfrente, permite modular el proceso de consolidación de la STM en LTM, modificando el nivel de relevancia de los eventos, pudiendo generar memorias muy fuertes y hábitos arraigados. Ejemplos de esto son los flashbacks y las memorias asociadas a eventos importantes.



\section{Motivación}

La memoria es una habilidad cognitiva crucial para los humanos. Al interactuar con otras personas o el ambiente les permite recordar experiencias pasadas y sus detalles. Luego, es de esperar que un robot de servicio posea una memoria que le permita potenciar sus capacidades de interacción con los humanos que ayudará\cite{Vijayakumar2014}. Una LTM permitiría, por ejemplo, generar diálogos interesantes sobre eventos pasados o cosas que el robot puede inferir del comportamiento humano, por otro lado, también permitiría la generalización de las tareas que tiene que llevar a cabo.

Particularmente, dado el enfoque de las plataformas disponibles, Bender cómo robot mayordomo y Pepper cómo robot social, se espera que ambos posean capacidades avanzadas de interacción con los humanos, para lo que se requiere una LTM.


\subsection{Problema}

El año 2015 se desarrolló una LTM episódica para el robot Bender, orientada a la interacción con personas y objetos\cite{Sanchez:2015}. El trabajo consideraba métodos para almacenar, adquirir y manejar la información episódica, sumado a un proceso simple de consolidación de memoria.

Actualmente la memoria desarrollada no está operativa, ni es factible habilitarla. A continuación se listan los aspectos que se consideran causas del problema desde un punto de vista técnico y humano:
\begin{itemize}
\item No se integró adecuadamente al software del robot, no se recopila ni provee información continuamente mientras el robot está en funcionamiento.
\item La memoria no provee una API que siga el estándar de los desarrollos del equipo, por lo que no se usa ni es mantenida.
\item RoboCup@Home no considera el uso de LTM en sus competencias, por lo que el equipo no tiene un incentivo real para seguir desarrollando o mantener la memoria. Esto además ha provocado que el código quede obsoleto.
\end{itemize}

Por otro lado, suponiendo que lo anterior estuviese solucionado, aún existen los siguientes problemas:
\begin{itemize}
\item Sólo considera 2 modelos semánticos: Persona y Objeto, para los cuales sólo se almacena información de nombre, nacionalidad e imagen.
\item A pesar de considerar un modelo para objetos, no se integró con los módulos relacionados que recopilan la información, por lo que realmente la memoria sólo funciona para entidades de tipo Persona.
\item Es esperable que una memoria considere más modelos (Personas, Objetos, Autos, Niños, Mascotas, etc) y más características para cada modelo (nombre, hobbies, trabajo, etc).
\item La consolidación de memoria STM a LTM sólo considera la primera interacción con cada entidad, por lo que no existe actualización de los datos.
\item Hay una restricción en los modelos y características a almacenar, respecto a la información que el robot es realmente capaz de obtener.
\end{itemize}


\subsection{Oportunidad}

Existe un vasto desarrollo respecto a la memoria y los procesos cognitivos, sin embargo, la investigación se concentra en campos como psicología, neurología y ciencias cognitivas\todo{cita aqui!}. Los estudios de LTM para robots de servicio son muy acotados y no existe una solución estándar a implementar. Algunos robots, como la versión comercial de Pepper, utilizan LTM, pero el código asociado no es libre, ni está basado en ROS.

El uso de LTM no está en las prioridades ``RoboCup'' del equipo, sino que es algo útil para demostraciones y para potenciar la interacción humano-robot. Por ello, se considera que no basta con desarrollar un módulo capaz de recopilar información inteligentemente, sino que además se requiere una integración con las capacidades de diálogo o de inferencia de información, para finalmente proveer una demostración de estas habilidades.

Así, esta es una oportunidad para diseñar una LTM para robots de servicio, que considere aspectos como: 
\begin{itemize}
\item Memoria episódica y semántica adecuada a tareas generales de robots de servicio.
\item Metodología para consolidación de STM en LTM.
\item Servicio para recopilación continua de información
\item Implementación estándar ROS, adecuada a las plataformas donde se implantará la solución.
%\item Capacidad de generar respaldos de la memoria y recuperación de éstos. 
\item Memoria emocional que permita dar relevancia a los eventos.
\item Inferencia de información a partir de datos de la memoria. Por ejemplo: ``Juan suele desayunar a las 9 am'', ``El control de la TV suele estar en el sofá'', etc.
%\item Integración con el diálogo que realiza el robot.\unsure{quitarlo?}
\end{itemize}

Tanto la memoria emocional, como la inferencia de información, se consideran requisitos deseables, por lo que están fuera del \textit{core} del proyecto.

%También se considera que es la oportunidad de promover la inclusión de desafios basados en LTM en la liga @Home, a partir de los resultados de este trabajo. Así, el desarrollo de LTMs y capacidades asociadas dejaría de ser postergado y pasaría a ser una prioridad para los equipos de la competencia.


%\subsection{Aporte del Trabajo}
%La contribución del trabajo es principalmente el diseño e implementación 
%\todo[inline]{Contribución del trabajo}

\section{Objetivos}

\subsection{Objetivo General}

El objetivo general corresponde al diseño de una LTM para robots de servicio domésticos, que considere componentes episódicos y semánticos. La LTM debe ser integrada en Bender, con un servicio en background que recopile información y con una API acorde a los desarrollos de UChile Homebreakers. Además, la LTM debe quedar integrada con alguna demostración de esta capacidad, ya sea mediante el diálogo o mediante inferencia de información.

En resumen, el producto final debe ser una LTM integrada en Bender, de generación continua de recuerdos y una demostración de su funcionamiento.


\subsection{Objetivos Específicos}

A continuación se presentan los objetivos específicos del trabajo, a modo de desglose del objetivo general en tareas más acotadas.

\begin{itemize}
\item Definición del proceso de consolidación de recuerdos.
\item Diseño de la arquitectura del sistema y validación
\item Implementación de la LTM y su API.
\item Servicio para recopilación continua de información.
\item Implementación de la demostración.
\end{itemize}

Otros objetivos específicos, correspondientes a requisitos que no son del \textit{core} del proyecto, son la implementación de la memoria emocional y la implementación de un módulo de inferencia de información basada en LTM.


%\section{Alcances}
\todo[inline]{Sobre los alcances del Trabajo}

%\section{Estructura de la memoria}
\todo[inline]{Sobre la estructura de la memoria}


\todo[inline]{Sobre la contribución específica del trabajo}
