
\chapter{Introducci\'on}

\section{Antecedentes Generales}

A continuaci\'on se presenta una breve introducci\'on a los temas requeridos para contextualizar este trabajo de t\'itulo: La rob\'otica de servicio y el equipo de trabajo donde se implantar\'a la soluci\'on. Adem\'as, se introduce el tema de la memoria humana, requerido para entender la propuesta y su relaci\'on con la rob\'otica.


\subsection{Robots de servicio}

La rob\'otica de servicio es un \'area enfocada en asistir a los seres humanos en tareas repetitivas y comunes, como la recolecci\'on de basura. Formalmente, se define un \textit{robot de servicio} como un robot ``que realiza tareas \'utiles para humanos o equipamiento, excluyendo aplicaciones de automatizaci\'on industrial''\cite{IFR}. Luego, el robot requiere cierto grado de autonom\'ia, que es la habilidad de actuar a partir del estado actual, usando lo que observa del ambiente y sin intervenci\'on humana. As\'i, un robot de servicio debe trabajar en ambientes no controlados y con la autonom\'ia suficiente que le permita llevar a cabo su cometido.

Un caso de uso t\'ipico es la asistencia en las tareas del hogar, donde se espera que un robot pueda ayudar a ordenar, preparar comida u ofrecer bebestibles. Otros casos de uso consideran el cuidado de adultos mayores, robots para compa\~nia en el hogar, mascotas robots, salud o educaci\'on. Particularmente, la compa\~nia SoftBank Robotics es pionera en ofrecer a Pepper como el primer robot humanoide ya adoptado en hogares de Jap\'on, as\'i como robot de bienvenida en hoteles y tiendas\cite{softbank}.


\subsection{Equipo de Trabajo: UChile Homebreakers}

El laboratorio de rob\'otica del Departamento de Ingenier\'ia El\'ectrica de la Universidad de Chile alberga dos equipos de rob\'otica: \textit{UChile Robotics Team}, dedicado al f\'utbol rob\'otico y \textit{UChile Homebreakers Team}, enfocado en rob\'otica de servicio. Ambos son conformados por alumnos de pregrado y postgrado de diversas especialidades, y liderados por el profesor Javier Ruiz del Solar\cite{uchile-robotics}.

UChile Homebreakers existe desde el a\~no 2007 y actualmente cuenta con 15 estudiantes. Todo su desarrollo de software est\'a basado en ROS, un framework para el desarrollo de plataformas rob\'oticas y con miles de usuarios alrededor del mundo\cite{ROS:2009}.

El equipo trabaja en dos plataformas humanoides, Bender y Pepper. Bender es un robot construido en el mismo laboratorio y con el objetivo de ser un mayordomo para el hogar. Pepper, desarrollado por SoftBank Robotics, est\'a dise\~nado para ser un robot de compa\~nia. Ambos comparten la misma arquitectura de software y pr\'acticamente todo su c\'odigo, exceptuando los drivers para acceder al hardware respectivo.


\subsubsection{RoboCup @Home League}


La RoboCup es una competencia internacional cuyo objetivo es ser un veh\'iculo para el desarrollo de la rob\'otica y la inteligencia artificial. Est\'a compuesta de variadas ligas: Rescue, Soccer, Simulation, @Home, Industrial y Junior, cada una con diversas subligas orientadas a fomentar la investigaci\'on de distintos aspectos del campo. Su sue\~no es que para mediados del siglo 21, un equipo de f\'utbol rob\'otico completamente aut\'onomo sea capaz de vencer al campi\'on de la \'ultima copa mundial y siguiendo las reglas de la FIFA\cite{robocup:rulebook_2017}.

UChile Homebreakers participa desde el a\~no 2007 en la categor\'ia @Home. Las pruebas de la liga se desarrollan en escenarios que imitan ambientes reales, como un hogar o un restaurante. 
%Adem\'as, la competencia funciona como un espect\'aculo para p\'ublico general, por lo que se priorizan pruebas y demostraciones interesantes para los espectadores.
% 
Las capacidades generalmente evaluadas y potenciadas en @Home son de Vision Computacional, Navegaci\'on aut\'onoma, Manipulaci\'on de objetos y Reconocimiento de Voz. Cada a\~no el equipo planifica sus desarrollos de acuerdo a los requerimientos de la competencia, por lo que trabajos fuera de las \'areas mencionadas no son considerados una prioridad.


\subsection{La memoria humana}

La memoria hace relaci\'on al almacenamiento de experiencias en el cerebro. Hay m\'ultiples sistemas de memoria independientes y sustentados por distintas estructuras cerebrales. A grandes rasgos, la memoria se puede dividir en de corto plazo STM (Short-Term Memory) y de largo plazo LTM (Large-Term Memory). La STM maneja informaci\'on muy detallada, es de poca capacidad y permite un r\'apido acceso, mientras que la LTM maneja mucha informaci\'on sobre experiencias y entidades, es menos detallada y de acceso m\'as lento\cite{Eichenbaum:2008}.

La LTM se puede dividir en expl\'icita (consciente) e impl\'icita (inconsciente). La primera almacena datos epis\'odicos, pudiendo responder las preguntas ``Qu\'e'', ``D\'onde'' y ``Cu\'ando'', datos sem\'anticos, que modelan hechos y conceptos como el lenguaje o personas, y tambi\'en, las conexiones entre ambas submemorias. La memoria impl\'icita codifica habilidades, h\'abitos y preferencias.

Existen procesos de consolidaci\'on y deterioro de la memoria que est\'an constantemente en funcionamiento. La consolidaci\'on requiere un est\'imulo relevante, sumado al proceso de almacenamiento, lo que genera conexiones entre la memoria epis\'odica y la respectiva zona sem\'antica. En caso de haber experiencias repetidas, las conexiones se fortalecen. El deterioro de la memoria es un proceso que degenera las conexiones entre ambas formas de memorias expl\'icitas.

La memoria emocional es una forma de memoria impl\'icita que genera reacciones emocionales y sentimientos. Seg\'un los est\'imulos a los que se enfrente, permite modular el proceso de consolidaci\'on de la STM en LTM, modificando el nivel de relevancia de los eventos, pudiendo generar memorias muy fuertes y h\'abitos arraigados. Ejemplos de esto son los flashbacks y las memorias asociadas a eventos importantes.



\section{Motivaci\'on}

La memoria es una habilidad cognitiva crucial para los humanos. Al interactuar con otras personas o el ambiente les permite recordar experiencias pasadas y sus detalles. Luego, es de esperar que un robot de servicio posea una memoria que le permita potenciar sus capacidades de interaci\'on con los humanos que ayudar\'a\cite{Vijayakumar2014}. Una LTM permitir\'ia, por ejemplo, generar di\'alogos interesantes sobre eventos pasados o cosas que el robot puede inferir del comportamiento humano, por otro lado, tambi\'en permitir\'ia la generalizaci\'on de las tareas que tiene que llevar a cabo.

Particularmente, dado el enfoque de las plataformas a utilizar, Bender c\'omo robot mayordomo y Pepper c\'omo robot social, se espera que ambos posean capacidades avanzadas de interaci\'on con los humanos, para lo que se requiere una LTM.


\subsection{Problema}

El a\~no 2015 se desarroll\'o una LTM epis\'odica para el robot Bender, orientada a la interacci\'on con personas y objetos\cite{Sanchez:2015}. El trabajo consideraba m\'etodos para almacenar, adquirir y manejar la informaci\'on epos\'odica, sumado a un proceso simple de consolidaci\'on de memoria.

Actualmente la memoria desarrollada no est\'a operativa, ni es factible habilitarla. A continuaci\'on se listan los aspectos que se consideran causas del problema desde un punto de vista t\'ecnico y humano:
\begin{itemize}
\item No se integr\'o adecuadamente al software del robot, no se recopila ni provee informaci\'on continuamente mientras el robot est\'a en funcionamiento.
\item La memoria no provee una API que siga el est\'andar de los desarrollos del equipo, por lo que no se usa ni es mantenida.
\item RoboCup@Home no considera el uso de LTM en sus competencias, por lo que el equipo no tiene un incentivo real para seguir desarrollando o mantener la memoria. Esto adem\'as ha provocado que el c\'odigo quede obsoleto.
\end{itemize}

Por otro lado, suponiendo que lo anterior estuviese solucionado, a\'un existen los siguientes problemas:
\begin{itemize}
\item S\'olo considera 2 modelos sem\'anticos: Persona y Objeto, para los cuales s\'olo se almacena informaci\'on de nombre, nacionalidad e imagen.
\item A pesar de considerar un modelo para objetos, no se integr\'o con los m\'odulos relacionados que recopilan la informaci\'on, por lo que realmente la memoria s\'olo funciona para entidades de tipo Persona.
\item Es esperable que una memoria considere m\'as modelos (Personas, Objetos, Autos, Ni\~nos, Mascotas, etc) y m\'as caracter\'isticas para cada modelo (nombre, hobbies, trabajo, etc).
\item La consolidaci\'on de memoria STM a LTM s\'olo considera la primera interacci\'on con cada entidad, por lo que no existe actualizaci\'on de los datos.
\item Hay una restricci\'on en los modelos y caracter\'isticas a almacenar, respecto a la informaci\'on que el robot es realmente capaz de obtener.
\end{itemize}


\subsection{Oportunidad}

Existe un vasto desarrollo respecto a la memoria y los procesos cognitivos, sin embargo, la investigaci\'on se concentra en campos como psicolog\'ia, neurolog\'ia y ciencias cognitivas. 
Los estudios de LTM para robots de servicio son muy acotados y no existe una soluci\'on est\'andar a implementar. Algunos robots, como la versi\'on comercial de Pepper, utilizan LTM, pero el c\'odigo asociado no es libre, ni est\'a basado en ROS.

El uso de LTM no est\'a en las prioridades ``RoboCup'' del equipo, sino que es algo \'util para demostraciones y para potenciar la interacci\'on humano-robot. Por ello, se considera que no basta con desarrollar un m\'odulo capaz de recopilar informaci\'on inteligentemente, sino que adem\'as se requiere una integraci\'on con las capacidades de di\'alogo o de inferencia de informaci\'on, para finalmente proveer una demostraci\'on de \'estas habilidades.

As\'i, \'esta es una oportunidad para dise\~nar una LTM para robots de servicio, que considere aspectos como: 
\begin{itemize}
\item Memoria epis\'odica y sem\'antica adecuada a tareas generales de robots de servicio.
\item M\'etodolog\'ia para consolidaci\'on de STM en LTM.
\item Servicio para recopilaci\'on continua de informaci\'on
\item Implementaci\'on est\'andar ROS, adecuada a las plataformas donde se implantar\'a la soluci\'on.
%\item Capacidad de generar respaldos de la memoria y recuperaci\'on de \'estos. 
\item Memoria emocional que permita dar relevancia a los eventos.
\item Inferencia de informaci\'on a partir de datos de la memoria. Por ejemplo: ``Juan suele desayunar a las 9am'', ``El control de la TV suele estar en el sof\'a'', etc.
%\item Integraci\'on con el di\'alogo que realiza el robot.\unsure{quitarlo?}
\end{itemize}

Tanto la memoria emocional, como la inferencia de informaci\'on, se consideran requisitos deseables, por lo que est\'an fuera del \textit{core} del proyecto.

%Tambi\'en se considera que es la oportunidad de promover la inclusi\'on de desafios basados en LTM en la liga @Home, a partir de los resultados de \'este trabajo. As\'i, el desarrollo de LTMs y capacidades asociadas dejar\'ia de ser postergado y pasar\'ia a ser una prioridad para los equipos de la competencia.


%\subsection{Aporte del Trabajo}
%La contribuci\'on del trabajo es principalmente el dise\~no e implementaci\'on 
\todo[inline]{Contribuci\'on del trabajo}

\section{Objetivos}

\subsection{Objetivo General}

El objetivo general corresponde al dise\~no de una LTM para robots de servicio, que considere componentes epis\'odicos y sem\'anticos. La LTM debe ser integrada en Bender y Pepper, con un servicio en background que recopile informaci\'on y con una API acorde a los desarrollos de UChile Homebreakers. Adem\'as, la LTM debe quedar integrada con alguna demostraci\'on de \'esta capacidad, ya sea mediante el di\'alogo o mediante inferencia de informaci\'on.

En resumen, el producto final debe ser una LTM integrada en los robots, de generaci\'on continua de recuerdos y que provea una demostraci\'on de \'esta capacidad.


\subsection{Objetivos Espec\'ificos}

A continuaci\'on se presentan los objetivos espec\'ificos del trabajo, a modo de desglose del objetivo general en tareas m\'as acotadas.

\begin{itemize}
\item Definici\'on del proceso de consolidaci\'on de recuerdos.
\item Dise\~no de la arquitectura del sistema y validaci\'on
\item Implementaci\'on de la LTM y su API.
\item Servicio para recopilaci\'on continua de informaci\'on.
\item Implementaci\'on de la demostraci\'on.
\end{itemize}

Otros objetivos espec\'ificos, correspondientes a requisitos que no son del \textit{core} del proyecto, son la implementaci\'on de la memoria emocional y la implementaci\'on de un m\'odulo de inferencia de informaci\'on basada en LTM.


%\section{Alcances}
\todo[inline]{ALCANCES}

%\section{Estructura de la memoria}
\todo[inline]{ESTRUCTURA DE LA MEMORIA1}
